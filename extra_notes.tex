\section{Poincar\'e Other NOTES/QUOTES}

\subsection{Poincar\'e on Correlative Movement (relation to Cybernetics)}

OUT OF SCOPE?

\begin{quote}
    Suppose a solid body to occupy successively the positions $\alpha$ and $\beta$; in the first position it will give us an aggregate of impressions $A$, and in the second position the aggregate of impressions $B$.  Now let there be a second solid body, of qualities entirely different from the first---of different colour, for instance.  Assume it to pass from the position $\alpha$, where it gives us the aggregate of impressions $A'$ to the position $\beta$, where it gives the aggregate of impressions $B'$.  In general, the aggregate $A$ will have nothing in common with the aggregate $A'$, nor will the aggregate $B$ have anything in common with the aggregate $B'$.  The transition from the aggregate $A$ to the aggregate $B$, and that of the aggregate $A'$ to the aggregate $B'$, are therefore two changes which \emph{in themselves} have in general nothing in common.  Yet we consider both these changes as displacements; and, further, we can consider them the \emph{same} displacement.  How can this be?  It is simply because they may be both corrected by the \emph{same} correlative movement of our body.  ``Correlative movement,'' therefore, constitutes the \emph{sole connection} between two phenomena which otherwise we should never have dreamed of connecting.  
    
    On the other hand, our body, thanks to the number of its articulations and muscles, may have a multitude of different movements, but all are not capable of ``correcting'' a modification of external objects; those alone are capable of it in which our whole body, or at least all those in which the organs of our senses enter into play are displaced \emph{en bloc---i.e.}, without any variation of their relative positions, as in the case of a solid body.  
    
    To sum up: 
    
    1.  In the first place, we distinguish two categories of phenomena: --- The first involuntary, unaccompanied by muscular sensations, and attributed to external objects---they are external changes; the second, of opposite character and attributed to the movements of our own body, are internal changes.
    
    2.  We notice that certain changes of each in these categories may be corrected by a correlative change of the other category.
    
    3.  We distinguish among external changes those that have a correlative in the other category---which we call displacements; and in the same way we distinguish among the internal changes those which have a correlative in the first category.
    
    Thus by means of this reciprocity is defined a particular class of phenomena called displacements.  \emph{The laws of these phenomena are the object of geometry.}
    
    \emph{Law of Homogeneity.}---The first of these laws is the law of homogeneity.  Suppose that by an external change we pass from the aggregate of impressions $A$ to the aggregate $B$, and that then this change $\alpha$ is corrected by a correlative voluntary movement $\beta$, so that we are brought back to the aggregate $A$.  Suppose now that another external change $\alpha'$ brings us again from the aggregate $A$ to the aggregate $B$.  Experiment then shows us that this change $\alpha'$, like the change $\alpha$, may be corrected by a voluntary correlative movement $\beta'$, and that this movement $\beta'$ corresponds to the same muscular sensations as the movement $\beta$ which corrected $\alpha$.
    
    This fact is usually enunciated as follows:---\emph{Space is homogeneous and isotropic.}  We may also say that a movement which is once produced may be repeated a second and a third time, and so on, without any variation of its properties.  In the first chapter, in which we discussed the nature of mathematical reasoning, we saw the importance that should be attached to the possibility of repeating the same operation indefinitely.  The virtue of mathematical reasoning is due to this repetition; by means of the law of homogeneity geometrical facts are apprehended.  To be complete, to the law of homogeneity must be added a multitude of other laws, into the details of which I do not propose to enter, but which mathematicians sum up by saying that these displacements form a ``group.''
    
    \citep[p. 61-64]{Poincare1952}
    
\end{quote}

In general Poincar\'e pursues the strategy of saying what geometry one \emph{would} come to use to characterize transformations in experience (or equivalently, transformations in points of view) if an idealized subject were embedded in a different space or world.  Ashby would perhaps be content with an effective list of state transitions, at least from the perspective of ``the system'', but if we are on a meta-level as scientists and philosophers, parsimonious summaries of these transitions, as found in one or another suitable system of geometry, are fine to construct.  So long as at the systems-level of analysis, it is realized that the organism is not ``comprehending'' or ``knowing'' a geometry, rather it has learned an adaptive set of transitions (which can be summarized by meta-cognitive agents).

One can also pursue the idealized-agent-embedded-in-a-world idea to consider how e.g. self-similarity on a fractal surface (created by one of Poincar\'e's much appealed to recurrent processes, and characterized e.g. by a Hausdorff dimension $d_H$) would come to be reflected in the agent's epistemology.  Not just the ``cognitive geometry'' but the \emph{rules} that can be relied upon.  These rules being, naturally, a deductive form of the rule of induction (both mathematical and scientific observation forms).  On the ``surface'' of such a world, an agent would come to characterize the world by a certain property at each point in his experienced world, e.g. $d_H$.  Much like how a property of curvature in Riemannian geometry is \emph{embedded} in a fundamental way, and reflected by us as experiencers of a world that is summed up in general relativity.

\begin{quote}
    The sense of light, even with one eye, together with the muscular sensations relative to the movements of the eyeball, will suffice to enable us to conceive of space of three dimensions.  The images of external objects are painted on the retina, which is a plane of two dimensions; these are \emph{perspectives}.  But as eye and objects are movable, we see in succession different perspectives of the same body taken from different points of view.  We find at the same time that the transition from one perspective to another is often acccompanied by muscular sensations.  If the transition from the perspective $A$ to the perspective $B$, and that of the perspective $A'$ to the perspective $B'$ are accompanied by the same muscular sensations, we connect them as we do other operations of the same nature.  Then when we study the laws according to which these operations are combined, we see that they form a group, which has the same structure as that of the movements of invariable solids.  Now, we have seen that it is from the properties of this group that we derive the idea of geometrical space and that of three dimensions.  We thus understand how these perspectives gave rise to the conception of three dimensions, although each perspective is of only two dimensions,---because \emph{they succeed each other according to certain laws.}  Well, in the same way that we draw the perspective of a three-dimensional figure on a plane, so we can draw that of a four-dimensional figure on a canvas of three (or two) dimensions.  To a geometer this is but child's play.  We can even draw several perspectives of the same figure from several different points of view.  We can easily represent to ourselves these perspectives, since they are of only three dimensions.  Imagine that the different perspectives of one and the same object to occur in succession, and that the transition from one to the other is accompanied by muscular sensations.  It is understood that we shall consider two of these transitions as two operations of the same nature when they are associated with the same muscular sensations.  There is nothing, then, to prevent us from imagining that these operations are combined according to any law we choose---for instance, by forming a group with the same structure as that of the movements of an invariable four-dimensional solid.  In this there is nothing that we cannot represent to ourselves, and, moreover, these sensations are those which a being would experience who has a retina of two dimensions, and who may be displaced in space of four dimensions.  In this sense we may say that we can represent to ourselves the fourth dimension.  
    
    \citep[p. 68-70]{Poincare1952}
\end{quote}

\begin{quote}
    The object of geometry is the study of a particular ``group''; but the general concept of group pre-exists in our minds, at least potentially.  It is imposed on us not as a form of our sensitiveness, but as a form of our understanding; only, from among all possible groups, we must choose one that will be the \emph{standard}, so to speak, to which we shall refer natural phenomena.
    
    Experiment guides us in this choice, which it does not impose on us.  It tells us not what is the truest, but what is the most convenient geometry.  It will be noticed that my description of these fantastic worlds has required no language other than that of ordinary geometry.  Then, were we transported to those worlds, there would be no need to change that language.  Beings educated there would no doubt find it more convenient to create a geometry different from ours, and better adapted to their impressions; but as for us, in the presence of the same impressions, it is certain that we should not find it more convenient to make a change.
    
    \citep[p. 70-71]{Poincare1952}
\end{quote}

Is he saying that we shouldn't change our internal agentic geometric conceptions from Euclidean to non-Euclidean?  How does this compare to a cybernetics/Ashby perspective?

-Interior Angles of Triangles in non-Euclidean geometries

\subsection{Pragmatic Poincar\'e}

-Geometrical systems as convention

-Measurement/Experiment not capable of being a crucial experiment to decide one geometrical system over another

-Chapter 5 point 6... isn't this wrong?

\begin{quote}
    6.  Experiments only teach us the relations of bodies to one another.  They do not and cannot give us the relations of bodies and space, nor the mutual relations of the different parts of space.  ``Yes!'' you reply, ``a single experiment is not enough, because it only gives us one equation with several unknowns; but when I have made enough experiments I shall have enough equations to calculate all my unknowns.''  If I know the height of the main-mast, that is not sufficient to enable me to calculate the age of the captain.  When you have measured every fragment of wood in a ship you will have many equations, but you will be no nearer knowing the captain's age.  All your measurements bearing on your fragments of wood can tell you only what concerns those fragments; and similarly, your experiments, however numerous they may be, referring only to the relations of bodies with one another, will tell you nothing about the mutual relations of the different parts of space.
    
    \citep[p. 79-80]{Poincare1952}
\end{quote}

Contrast this with e.g. what Harris says is implied by quantum and relativity.  

-p. 84 "Hence, \emph{experiments have reference not to space but to bodies.}

- Reread section 8 in chapter 5, more relationship to Ashby?

\begin{quote}
    [Kirchoff] wanted a definition of a force, and he took the first that came handy; but we do not require a definition of force; the idea of force is primitive, irreducible, indefinable; we all know what it is; of it we have direct intuition.  This direct intuition arises from the idea of effort which is familiar to us from childhood.  But in the first place, even if this direct intuition made known to us the real nature of force in itself, it would prove to be an insufficient basis for mechanics; it would, moreover, be quite useless.  The important thing is not to know what force is, but how to measure it.  Everything which does not teach us how to measure it is as useless to the mechanician as, for instance, the subjective idea of heat and cold to the student of heat.  
    \citep[p. 105-106]{Poincare1952}
\end{quote}

Intuitionism (about physical concepts) into pragmatics about measurement.

- Contrary hypothesis to relativity (principle of relative motion) would be ``repugnant to the mind.'' \citep[p. 111, 113]{Poincare1952}

- Physical laws, not just geometry, as expressions of convenience to satisfy the mind. (e.g. p. 117)


- Energetics (conservation of energy kinetic + potential energy, and Hamilton's principle of least action) \href{https://en.wikipedia.org/wiki/Hamilton%27s_principle}{wiki}

\begin{quote}
    Every change that the bodies of nature can undergo is regulated by two experimental laws.  First, the sum of the kinetic and potential energies is constant.  This is the principle of the conservation of energy.  Second, if a system of bodies is at $A$ at the time $t_0$, and at $B$ at the time $t_1$, it always passes from the first position to the second by such a path that the \emph{mean} value of the difference between the two kinds of energy in the interval of time which separates the two epochs $t_0$ and $t_1$ is a minimum.  This is Hamilton's principle, and is one of the forms of the principle of least action.  
    
    The energetic theory has the following advantages over the classical.  First, it is less incomplete---that is to say, the principles of the conservation of energy and of Hamilton teach us more than the fundamental principles of the classical theory, and exclude certain motions which do not occur in nature and which would be compatible with the classical theory.  Second, it frees us from the hypothesis of atoms, which it was almost impossible to avoid with the classical theory.  \citep[p. 123-124]{Poincare1952}
 \end{quote}
 
 \begin{quote}
     If $T + U + Q$ were of the particular form that I have suggested above, no ambiguity would ensue.  Among the functions $\phi(T + U + Q)$ which remain constant, there is only one that would be of this particular form, namely the one which I would agree to call energy.  But I have said this is not rigorously the case.  Among the functions that remain constant there is not one which can rigorously be placed in this particular form.  How then can we choose from among them that which should be called energy?  We have no longer any guide in our choice.
     
     Of the principle of conservation of energy there is nothing left then but an enunciation:---\emph{There is something which remains constant.}  In this form it, in its turn, is outside the bounds of experiment and reduced to a kind of tautology.  It is clear that if the world is governed by laws there will be quantities which remain constant.  Like Newton's laws, and for an analogous reason, the principle of the conservation of energy being based on experiment, can no longer be invalidated by it.  \citep[p. 127-128]{Poincare1952}
 \end{quote}
 
 - law of errors reference: \href{https://en.wikipedia.org/wiki/Laplace_distribution}{wiki}
 
 \begin{quote}
     An eminent physicist said to me one day, \emph{\'apropos} of the law of errors:---every one stoutly believes it, because mathematicians imagine that it is an effect of observation, and observers imagine that it is a mathematical theorem.  And this was for a long time the case with the principle of the conservation of energy.  It is no longer the same now.  There is no one who does not know that it is an experimental fact.  But then who gives us the right of attributing to the principle itself more generality and more precision than to the experiments which have served to demonstrate it?  This is asking, if it is legitimate to generalise, as we do every day, empiric data, and I shall not be so foolhardy as to discuss this question, after so many philosophers have vainly tried to solve it.  \citep[p. 129]{Poincare1952}
 \end{quote}
 
 - Is he referring to e.g. Hume?  Yes I think so given subsequent paragraphs.
 
 - Mayer's law (thermodynamic conservation energy) \href{https://en.wikipedia.org/wiki/Julius_von_Mayer}{wiki}

 - This relates to open system discussion ala Hartmann and Cuffaro
 
 \begin{quote}
     If the future state of the system is not entirely determined by its present state, it is because it further depends on the state of bodies external to the system.  But then, is it likely that there exist among the parameters $x$ which define the state of the system of equations independent of this state of the external bodies? and if in certain cases we think we can find them, is it not only because of our ignorance, and because the influence of these bodies is too weak for our experiment to be able to detect it?  If the system is not regarded as completely isolated, it is probable that the rigorously exact expression of its internal energy will depend upon the state of the external bodies.  Again, I have supposed above that the sum of all the external work is zero, and if we wish to be free from this rather artificial restriction the enunciation becomes still more difficult.  To formulate Mayer's principle by giving it an absolute meaning, we must extend it to the whole universe, and then we find ourselves face to face with the very difficulty we have endeavoured to avoid.  To sum up, and to use ordinary language, the law of the conservation of energy can have only one significance, because there is in it a property common to all possible properties; but in the determinist hypothesis there is only one possible, and then the law has no meaning.  In the indeterminist hypothesis, on the other hand, it would have a meaning even if we wished to regard it in an absolute sense.  It would appear as a limitation imposed on freedom.  \cite[p. 133-134]{Poincare1952}
 \end{quote}
 
 - Clausius's principle \href{https://en.wikipedia.org/wiki/Clausius_theorem}{wiki}
 

 
 \begin{quote}
     Almost everything that I have just said applies to the principle of Clausius.  What distinguishes it is, that it is expressed by an inequality.  It will be said perhaps that it is the same with all physical laws, since their precision is always limited by errors of observation.  But they at least claim to be first approximations, and we hope to replace them little by little by more exact laws.  If, on the other hand, the principle of Clausius reduces to an inequality, this is not caused by the imperfection of our means of observation, but by the very nature of the question.  \citep[p. 135]{Poincare1952}
 \end{quote}
 
  - measurement/observation inequalities like uncertainty principle/fourier transform
  
  In summarizing section III of the book and conventionalism:
 
 \begin{quote}
     The principles of mechanics are therefore presented to us under two different aspects.  On the one hand, there are truths founded on experiment, and verified approximately as far as almost isolated systems are concerned; on the other hand, there are postulates applicable to the whole of the universe and regarded as rigorously true.  If these postulates possess a generality and a certainty which falsify the experimental truths from which they were deduced, it is because they reduce in final analysis to a simple convention that we have a right to make, because we are certain beforehand that no experiment can contradict it.  This convention, however, is not absolutely arbitrary; it is not the child of our caprice.  We admit it because certain experiments have shown us that it will be convenient, and thus is explained how experiment has built up the principles of mechanics, and why, moreover, it cannot reverse them.  
     Take a comparison with geometry.  The fundamental propositions of geometry, for instance, Euclid's postulate, are only conventions, and it is quite as unreasonable to ask if they are true or false as to ask if the metric system is true or false.  Only, these conventions are convenient, and there are certain experiments which prove it to us.  At the first glance, the analogy is complete, the role of experiment seems the same.  We shall therefore be tempted to say, either mechanics must be looked upon as experimental science and then it should be the same with geometry; or, on the contrary, geometry is a deductive science, and then we can say the same of mechanics.  Such a conclusion would be illegitimate.  The experiments which have led us to adopt as more convenient the fundamental conventions of geometry refer to bodies which have nothing in common with those that are studied by geometry.  They refer to the properties of solid bodies and to the propagation of light in a straight line.  These are mechanical, optical experiments.  
     
     In no way can they be regarded as geometrical experiments.  And even the probable reason why our geometry seems convenient to us is, that our bodies, our hands, and our limbs enjoy the properties of solid bodies.  Our fundamental experiments are pre-eminently physiological experiments which refer, not to the space which is the object that geometry must study, but to our body---that is to say, to the instrument which we use for that study.  On the other hand, the fundamental conventions of mechanics and the experiments which prove to us that they are convenient, certainly refer to the same objects or to analogous objects.  Conventional and general principles are the natural and direct generalisations of experimental and particular principles.  \citep[p. 135-137]{Poincare1952}
 \end{quote}
 
 \begin{quote}
     We now understand why the teaching of mechanics should remain experimental.  Thus only can we be made to understand the genesis of the science, and that is indispensable for a complete knowledge of the science itself.  Besides, if we study mechanics, it is in order to apply it; and we can only apply it if it remains objective.  Now, as we have seen, when principles gain in generality and certainty they lose in objectivity.  It is therefore especially with the objective side of principles that we must be early familiarised, and this can only be by passing from the particular to the general, instead of from the general to the particular.
     
     Principles are conventions and definitions in disguise.  They are, however, deduced from experimental laws, and these laws have, so to speak, been erected into principles to which our mind attributes an absolute value.  Some philosophers have generalised far too much.  They have thought that the principles were the whole of science, and therefore that the whole of science was conventional.  This paradoxical doctrine, which is called Nominalism, cannot stand examination.  How can a law become a principle?  It expressed a relation between two real terms, $A$ and $B$; but it was not rigorously true, it was only approximate.  We introduce arbitrarily an intermediate term, $C$, more or less imaginary, and $C$ is \emph{by definition} that which has with $A$ \emph{exactly} the relation expressed by the law.  So our law is decomposed into an absolute and rigorous principle which expresses the relation of $A$ to $C$, and an approximate experimental and revisable law which expresses the relation of $C$ to $B$.  But it is clear that however far this decomposition may be carried, laws will always remain.  We shall now enter into the domain of laws properly so called.  \citep[p. 138-139]{Poincare1952}
 \end{quote}
 
 - conventionalism about geometry but not mechanics, and if fundamental mechanical laws are strictly false it is in a justified way different from the justifications for choosing a particular geometrical view applied to reality
 
 -similar to Cartwright's anti-realism about fundamental laws (because they contradict the augmented-with-assumption phenomenological laws)?
 
 
 \begin{quote}
     What, then, is a good experiment?  It is that which teaches us something more than an isolated fact.  It is that which enables us to predict, and to generalise.  Without generalisation, prediction is impossible.  The circumstances under which one has operated will never again be reproduced simultaneously.  The fact observed will never be repeated.  All that can be affirmed is that under analogous circumstances an analogous fact will be produced.  To predict it, we must therefore invoke the aid of analogy---that is to say, even at this stage, we must generalise.  However timid we may be, there must be interpolation.  Experiment only gives us a certain number of isolated points.  They must be connected by a continuous line, and this is a true generalisation.  But more is done.  The curve thus traced will pass between and near the points observed; it will not pass through the points themselves.  Thus we are not restricted to generalising our experiment, we correct it; and the physicist who would abstain from these corrections, and really content himself with experiment pure and simple, would be compelled to enunciate very extraordinary laws indeed.  Detached facts cannot therefore satisfy us, and that is why our science must be ordered, or, better still, generalised.
     
     It is often said that experiments should be made without preconceived ideas.  That is impossible.  Not only would it make every experiment fruitless, but even if we wished to do so, it could not be done.  Every man has his own conception of the world, and this he cannot so easily lay aside.  \citep[p. 142-143]{Poincare1952}
 \end{quote}
 
 ``It is far better to predict without certainty, than never to have predicted at all.'' \citep[p. 144]{Poincare1952}
 
 
 - mathematical physics directs generalisation (catalogue), analogy with library planning/expenditures
 
 - unity and simplicity.  simplicity/complexity process loop
 
 \begin{quote}
     No doubt, if our means of investigation became more and more penetrating, we should discover the simple beneath the complex, and then the complex from the simple, and then again the simple beneath the complex, and so on, without ever being able to predict what the last term will be.  We must stop somewhere, and for science to be possible we must stop where we have found simplicity.  That is the only ground on which we can erect the edifice of our generalisations.  But, this simplicity being only apparent, will the ground be solid enough?  That is what we have now to discover.  \citep[p. 148-149]{Poincare1952}
 \end{quote}
 
 -role of hypothesis, should be ASAP submitted to experimental verification, unconscious hypotheses dangerous, D/Q-like thesis and underdetermination
 
 \begin{quote}
     Let us also notice that it is important not to multiply hypotheses indefinitely.  If we construct a theory based upon multiple hypotheses, and if experiment condemns it, which of the premisses must be changed?  It is impossible to tell.  Conversely, if the experiment succeeds, must we suppose that it has verified all these hypotheses at once?  Can several unknowns be determined from a single equation?  \citep[p. 151-152]{Poincare1952}
 \end{quote}
 
 - also makes points similar to batterman?  153-157
 
 - optics
 
 \begin{quote}
     The mind has to run ahead of the experiment, and if it has done so with success, it is because it has allowed itself to be guided by the instinct of simplicity.  The knowledge of the elementary fact enables us to state the problem in the form of an equation.  It only remains to deduce from it by combination the observable and verifiable complex fact.  That is what we call \emph{integration}, and it is the province of the mathematician.  It might be asked, why in physical science generalisation so readily takes the mathematical form.  The reason is now easy to see.  It is not only because we have to express numerical laws; it is because the observable phenomenon is due to the superposition of a large number of elementary phenomena which are \emph{all similar to each other}; and in this way differential equations are quite naturally introduced.  It is not enough that each elementary phenomenon should obey simple laws: all those that we have to combine must obey the same law; then only is the intervention of mathematics of any use.  Mathematics teaches us, in fact, to combine like with like.  Its object is to divine the result of a combination without having to reconstruct that combination element by element.  If we have to repeat the same operation several times, mathematics enables us to avoid this repetition by telling the result beforehand by a kind of induction.  This I have explained before in the chapter on mathematical reasoning.  But for that purpose all these operations must be similar; in the contrary case we must evidently make up our minds to working them out in full one after the other, and mathematics will be useless.  It is therefore, thanks to the approximate homogeneity of the matter studied by physicists, that mathematical physics came into existence.  In the natural sciences the following conditions are no longer to be found:---homogeneity, relative independence of remote parts, simplicity of the elementary fact; and that is why the student of natural science is compelled to have recourse to other modes of generalisation.  \citep[p. 158-159]{Poincare1952}
 \end{quote}
 
 - associationist structure building?
 
 - ch. X The Theories of Modern Physics goes through examples where relations hold but objects we picture the relations holding change.  E.g. Fresnel, Coloumb, Carnot [structuralism]
 
 \begin{quote}
     No theory seemed established on firmer ground than Fresnel's, which attributed light to the movements of the ether.  Then if Maxwell's theory is today preferred, does that mean that Fresnel's work was in vain?  No; for Fresnel's object was not to know whether there really is an ether, if it is or is not formed of atoms, if these atoms really move in this way or that; his object was to predict optical phenomena.
     
     This Fresnel's theory enables us to do today as well as it did before Maxwell's time.  The differential equations are always true, they may be always integrated by the same methods, and the results of this integration still preserve their value.  It cannot be said that this is reducing physical theories to simple practical recipes; these equations express relations, and if the equations remain true, it is because the relations preserve their reality.  They teach us now, as they did then, that there is such and such a relation between this thing and that; only, the something which we then called \emph{motion}, we now call \emph{electric current}.  But these are merely names of the images we substituted for the real objects which Nature will hide for ever from our eyes.  The true relations between these real objects are the only reality we can attain, and the sole condition is that the same relations shall exist between these objects as between the images we are forced to put in their place.  \citep[p. 160-161]{Poincare1952}
 \end{quote}
 
 -when contradiction between two theories ``Let us not be troubled, but let us hold fast to the two ends of the chain, lest we lose the intermediate links.''
 
 \begin{quote}
     In case of contradiction one of them at least should be considered false.  But this is no longer the case if we only seek in them what should be sought.  It is quite possible that they both express true relations, and that the contradictions only exist in the images we have formed to ourselves of reality.  \citep[p. 163]{Poincare1952}
 \end{quote}
 
  - satisfying the mind; structuralism accounts for when old theories get brought back to scientific attention (Coulomb fluid example)
 
 \begin{quote}
     Hypotheses of this kind have therefore only a metaphorical sense.  The scientist should no more banish them than a poet banishes metaphor; but he ought to know what they are worth.  They may be useful to give satisfaction to the mind, and they will do no harm as long as they are only indifferent hypotheses.  
     
     These considerations explain to us why certain theories, that were thought to be abandoned and definitively condemned by experiment, are suddenly revived from their ashes and begin a new life.  It is because they expressed true relations, and had not ceased to do so when for some reason or other we felt it necessary to enunciate the same relations in another language.  Their life had been latent, as it were.  
     
     Barely fifteen years ago, was there anything more ridiculous, more quaintly old-fashioned, than the fluids of Coulomb?  And yet, here they are re-appearing under the name of \emph{electrons}.  In what do these permanently electrified molecules differ from the electric molecules of Coulomb?  It is true that in the electrons the electricity is supported by a little, a very little matter; in other words, they have mass.  Yet Coulomb did not deny mass to his fluids, or if he did, it was with reluctance.  It would be rash to affirm that the belief in electrons will not also undergo an eclipse, but it was none the less curious to note this unexpected renaissance.
     
     \citep[p. 164-165]{Poincare1952}
 \end{quote}
 
- Carnot second law example
 
 - pragmatic usefulness criteria
 
 - mechanisms, Hertz
 
 \begin{quote}
     Every time that the principles of least action and energy are satisfied, we shall see that not only is there always a mechanical explanation possible, but that there is an unlimited number of such explanations.  By means of a well-known theorem due to K\"onigs, it may be shown that we can explain everything in an unlimited number of ways, by connections after the manner of Hertz, or, again, by central forces.  \citep[p. 167-168]{Poincare1952}
 \end{quote}
 
 - ether
 
 \begin{quote}
     We may conceive of ordinary matter as either composed of atoms, whose internal movements escape us, our senses being able to estimate only the displacement of the whole; or we may imagine one of those subtle fluids, which under the name of \emph{ether} or other names, have from all time played so important a role in physical theories.  Often we go further, and regard the ether as the only primitive, or even as the only true matter.  The more moderate consider ordinary matter to be condensed ether, and there is nothing startling in this conception; but others only reduce its importance still further, and see in matter nothing more than the geometrical locus of singularities in the ether.  
     
     Lord Kelvin, for instance, holds what we call matter to be only the locus of those points at which the ether is animated by vortex motions.  Riemann believes it to be locus of those points at which ether is constantly destroyed; to Wiechert or Larmor, it is the locus of the points at which the ether has undergone a kind of torsion of a very particular kind.  Taking any one of these points of view, I ask by what right do we apply to the ether the mechanical properties observed in ordinary matter, which is but false matter?  The ancient fluids, caloric, electricity, etc., were abandoned when it was seen that heat is not indestructible.  But they were also laid aside for another reason.  In materialising them, their individuality was, so to speak, emphasised---gaps were opened between them; and these gaps had to be filled in when the sentiment of the unity of Nature became stronger, and when the intimate relations which connect all the parts were perceived.  In multiplying the fluids, not only did the ancient physicists create unnecessary entities, but they destroyed real ties.  It is not enough for a theory not to affirm false relations; it must not conceal true relations.  
     
     Does our ether actually exist?  We know the origin of our belief in the ether.  If light takes several years to reach us from a distant star, it is no longer on the star, nor is it on the earth.  It must be somewhere, and supported, so to speak, by some material agency.
     
     The same idea may be expressed in a more mathematical and more abstract form.  What we note are the changes undergone by the material molecules.  We see, for instance, that the photographic plate experiences the consequences of a phenomenon of which the incandescent mass of a star was the scene several years before.  Now, in ordinary mechanics, the state of the system under consideration depends only on its state at the moment immediately preceding; the system therefore satisfies certain differential equations.  On the other hand, if we did not believe in the ether, the state of the material universe would depend not only on the state immediately preceding, but also on much older states; the system would satisfy equations of finite differences.  The ether was invented to escape this breaking down of the laws of general mechanics.  
     
     Still, this would only compel us to fill the interplanetary space with ether, but not to make it penetrate into the midst of the material media.  Fizeau's experiment goes further.  By the interference of rays which have passed through the air or water in motion, it seems to show us two different media penetrating each other, and yet being displaced with respect to each other.  The ether is all but in our grasp.  Experiments can be conceived in which we come closer still to it.  Assume that Newton's principle of the equality of action and re-action is not true if applied to matter \emph{alone}, and that this can be proved.  The geometrical sum of all the forces applied to all the molecules would no longer be zero.  If we did not wish to change the whole of the science of mechanics, we should have to introduce the ether, in order that the action which matter apparently undergoes should be counterbalanced by the re-action of matter on something.
     
     Or again, suppose we discover that optical and electrical phenomena are influenced by the motion of the earth.  It would follow that those phenomena might reveal to us not only the relative motion of material bodies, but also what would seem to be their absolute motion.  Again, it would be necessary to have an ether in order that these so-called absolute movements should not be their displacements with respect to empty space, but with respect to something concrete.  
     
     Will this ever be accomplished?  I do not think so, and I shall explain why; and yet, it is not absurd, for others have entertained this view.  For instance, if the theory of Lorentz, or which I shall speak in more detail in Chapter XIII., were true, Newton's principle would not apply to matter \emph{alone}, and the difference would not be very far from being within reach of experiment.  On the other hand, many experiments have been made on the influence of the motion of the earth.  The results have always been negative.  But if these experiments have been undertaken, it is because we have not been certain beforehand; and indeed, according to current theories, the compensation would be only approximate, and we might expect to find accurate methods giving positive results.  I think that such a hope is illusory; it was none the less interesting to show that a success of this kind would, in a certain sense, open to us a new world.
     
     And now allow me to make a digression; I must explain why I do not believe, in spite of Lorentz, that more exact observations will ever make evident anything else but the relative displacements of material bodies.  Experiments have been made that should have disclosed the terms of the first order; the results were nugatory.  Could that have been by chance?  No one has admitted this; a general explanation was sought, and Lorentz found it.  He showed that the terms of the first order should cancel each other, but not the terms of the second order.  Then more exact experiments were made, which were also negative; neither could this be the result of chance.  An explanation was necessary, and was forthcoming; they always are; hypotheses are what we lack the least.  But this is not enough.  Who is there who does not think that this leaves to chance far too important a role?  Would it not also be a chance that this singular concurrence should cause a certain circumstance to destroy the terms of the first order, and that a totally different but very opportune circumstance should cause those of the second order to vanish?  No; the same explanation must be found for the two cases, and everything tends to show that this explanation would serve equally well for the terms of the higher order, and that the mutual destruction of these terms will be rigorous and absolute.
     
     \citep[p. 168-172]{Poincare1952}
     
 \end{quote}
 
 -unification of light, electricity, and magnetism.  Lorentz ether theory/electrons, mechanical explanation, analogies between systems
 
 \begin{quote}
     The most satisfactory theory is that of Lorentz; it is unquestionably the theory that best explains the known facts, the one that throws into relief the greatest number of known relations, the one in which we find most traces of definitive construction.  That it still possesses a serious fault I have shown above.  It is in contradiction with Newton's law that action and re-action are equal and opposite---or rather, this principle according to Lorentz cannot be applicable to matter alone; if it be true, it must take into account the action of the ether on matter, and the re-action of the matter on the ether.  Now, in the new order, it is very likely that things do not happen in this way.  
     
     However this may be, it is due to Lorentz that the results of Fizeau on the optics of moving bodies, the laws of normal and abnormal dispersion and of absorption are connected with each other and with the other properties of the ether, by bonds which no doubt will not be readily severed.  Look at the ease with which the new Zeeman phenomenon found its place, and even aided the classification of Faraday's magnetic rotation, which had defied all Maxwell's efforts.  This facility proves that Lorentz's theory is not a mere artificial combination which must eventually find its solvent.  It will probably have to be modified, but not destroyed.  
     The only object of Lorentz was to include in a single whole all the optics and electro-dynamics of moving bodies; he did not claim to give a mechanical explanation.  Larmor goes further; keeping the essential part of Lorentz's theory, he grafts upon it, so to speak, MacCullagh's ideas on the direction of the movement of the ether.  MacCullagh held that the velocity of the ether is the same in magnitude and direction as the magnetic force.  Ingenious as is this attempt, the fault in Lorentz's theory remains, and is even aggravated.  According to Lorentz, we do not know what the movements of the ether are; and because we do not know this, we may suppose them to be movements compensating those of matter, and re-affirming that action and re-action are equal and opposite.  According to Larmor we know the movements of the ether, and we can prove that the compensation does not take place.  
     
     If Larmor has failed, as in my opinion he has, does it necessarily follow that a mechanical explanation is impossible?  Far from it.  I said above that as long as a phenomenon obeys the two principles of energy and least action, so long it allows of an unlimited number of mechanical explanations.  And so with the phenomomena of optics and electricity.
     
     But this is not enough.  For a mechanical explanation to be good it must be simple; to choose it from among all the explanations that are possible there must be other reasons than the necessity of making a choice.  Well, we have no theory as yet which will satisfy this condition and consequently be of any use.  Are we then to complain?  That would be to forget the end we seek, which is not the mechanism; the true and only aim is unity.  
     
     We ought therefore to set some limits to our ambition.  Let us not seek to formulate a mechanical explanation; let us be content to show that we can always find one if we wish.  In this we have succeeded.  The principle of the conservation of energy has always been confirmed, and now it has a fellow in the principle of least action, stated in the form appropriate to physics.  This has also been verified, at least as far as concerns the reversible phenomena which obey Lagrange's equations---in other words, which obey the most general laws of physics.  The irreversible phenomena are much more difficult to bring into line; but they, too, are being coordinated and tend to come into the unity.  The light which illuminates them comes from Carnot's principle.  For a long time thermodynamics was confined to the study of the dilatations of bodies and of their change of state.  For some time past it has been growing bolder, and has considerably extended its domain.  We owe to it the theories of the voltaic cell and of their thermo-electric phenomena; there is not a corner in physics which it has not explored, and it has even attacked chemistry itself.  The same laws hold good; everywhere, disguised in some form or other, we find Carnot's principle; everywhere also appears that eminently abstract concept of entropy which is as universal as the concept of energy, and like it, seems to conceal a reality.  It seemed that radiant heat must escape, but recently that, too, has been brought under the same laws.  
     
     In this way fresh analogies are revealed which may be often pursued in detail; electric resistance resembles the viscosity of fluids; hysteresis would rather be like the friction of solids.  In all cases friction appears to be the type most imitated by the most diverse irreversible phenomena, and this relationship is real and profound.  
     \citep[p. 175 - 178]{Poincare1952}
 \end{quote}
 
 - physical chemistry and materials, unity though complex, satisfying mind again
 
 \begin{quote}
     As we get to know the properties of matter better we see that continuity reigns.  From the work of Andrews and Van der Waals, we see how the transition from the liquid to the gaseous state is made, and that it is not abrupt.  Similarly, there is no gap between the liquid and solid states, and in the proceedings of a recent Congress we see memoirs on the rigidity of liquids side by side with papers on the flow of solids.  
     
     With this tendency there is no doubt a loss of simplicity.  Such and such an effect was represented by straight lines; it is now necessary to connect these lines by more or less complicated curves.  On the other hand, unity is gained.  Separate categories quieted but did not satisfy the mind.  
     
     \citep[p. 181-182]{Poincare1952}
 \end{quote}
 
 - probability, subjective vs objective, gamblers fallacy, posterior and priors in inferring causes, base rate, curve fitting, 
 
 - a priori considerations for curves that are continuous and not sinuosities
 
 - law of errors (Gauss) as `transcendental curve', but for Poincare if it is transcendental it is because of convention and practicality that it is so.
 
 \begin{quote}
     We must calculate the probable \emph{a posteriori} value of each error, and therefore the probable value of the quantity to be measured.  But, as I have just explained, we cannot undertake this calculation unless we admit \emph{a priori--i.e.}, before any observations are made---that there is a law of the probability of errors.  Is there a law of errors?  The law to which all calculators assent is Gauss's law, that is represented by a certain transcendental curve known as the ``bell''.
     
     But it is first of all necessary to recall the classic distinction between systematic and accidental errors.  If the metre with which we measure a length is too long, the number we get will be too small, and it will be no use to measure several times---that is a systematic error.  If we measure with an accurate metre, we may make a mistake, and find the length sometimes too large and sometimes too small, and when we take the mean of a large number of measurements, the error will tend to grow small.  These are accidental errors.  
     
     It is clear that systematic errors do not satisfy Gauss's law, but do accidental errors satisfy it?  Numerous proofs have been attempted, almost all of them crude paralogisms.  But starting from the following hypotheses we may prove Gauss's law: the error is the result of a very large number of partial and independent errors; each partial error is very small and obeys any law of probability whatever, provided the probability of a positive error is the same as that of an equal negative error.  It is clear that these conditions will be often, but not always, fulfilled, and we may reserve the name of accidental for errors which satisfy them.
     
     We see that the method of least squares is not legitimate in every case; in general, physicists are more distrustful of it than astronomers.  This is no doubt because the latter, apart from the systematic errors to which they and the physicists are subject alike, have to contend with an extremely important source of error which is entirely accidental---I mean atmospheric undulations.  So it is very curious to hear a discussion between a physicist and an astronomer about a method of observation.  The physicist, persuaded that one good measurement is worth more than many bad ones, is pre-eminently concerned with the elimination by means of every precaution of the final systematic errors; the astronomer retorts: ``But you can only observe a small number of stars, and accidental errors will not disappear.''
     
     What conclusion must we draw?  Must we continue to use the method of least squares?  We must distinguish.  We have eliminated all the systematic errors of which we have any suspicion; we are quite certain that there are others still, but we cannot detect them; and yet we must make up our minds and adopt a definitive value which will be regarded as the probable value; and for that purpose it is clear that the best thing we can do is to apply Gauss's law.  We have only applied a practical rule referring to subjective probability.  And there is no more to be said.
     
     Yet we want to go farther and say that not only the probable value is so much, but that the probable error in the result is so much.  \emph{This is absolutely invalid:}  it would be true only if we were sure that all the systematic errors were eliminated, and of that we know absolutely nothing.  We have two series of observations; by applying the law of least squares we find that the probable error in the first series is twice as small as in the second.  The second series may, however, be more accurate than the first, because the first is perhaps affected by a large systematic error.  All that we can say is, that the first series is \emph{probably} better than the second because its accidental error is smaller, and that we have no reason for affirming that the systematic error is greater for one of the series than for the other, our ignorance on this point being absolute.
     
     \cite[p. 207-209]{Poincare1952}
 \end{quote}
 
 
 - ch 12 Optics and electricity, Fresnel.  THIS IS A VERY GOOD CHAPTER
 
\begin{quote}
    It is owing to Fresnel that the science of optics is more advanced than any other branch of physics.  The theory called the theory of undulations forms a complete whole, which is satisfying to the mind; but we must not ask from it what it cannot give us.  The object of mathematical theories is not to reveal to us the real nature of things; that would be an unreasonable claim.  Their only object is to co-ordinate the physical laws with which physical experiment makes us acquainted, the enunciation of which, without the aid of mathematics, we should be unable to effect.  Whether the ether exists or not matters little---let us leave that to the metaphysicians; what is essential for us is, that everything happens as if it existed, and that this hypothesis is found to be suitable for the explanation of phenomena.  After all, have we any other reason for believing in the existence of material objects?  That, too, is only a convenient hypothesis; only, it will never cease to be so, while some day, no doubt, the ether will be thrown aside as useless.
    
    \citep[p. 211-212]{Poincare1952}
\end{quote}

- Maxwell's theory

\begin{quote}
    The first time a French reader opens Maxwell's book, his admiration is tempered with a feeling of uneasiness, and often of distrust.
    
    It is only after prolonged study, and at the cost of much effort, that this feeling disappears.  Some minds of high calibre never lose this feeling.  Why is it so difficult for the ideas of this English scientist to become acclimatised among us?  No doubt the education received by most enlightened Frenchmen predisposes them to appreciate precision and logic more than any other qualities.  In this respect the old theories of mathematical physics gave us complete satisfaction.  All our masters, from Laplace to Cauchy, proceeded along the same lines.  Starting with clearly enunciated hypotheses, they deduced from them all their consequences with mathematical rigour, and then compared them with experiment.  It seemed to be their aim to give to each of the branches of physics the same precision as to celestial mechanics.  
    
    A mind accustomed to admire such models is not easily satisfied with a theory.  Not only will it not tolerate the least appearance of contradiction, but it will expect the different parts to be logically connected with one another, and will require the number of hypotheses to be reduced to a minimum.
    
    This is not all; there will be other demands which appear to me to be less reasonable.  Behind the matter of which our senses are aware, and which is made known to us by experiment, such a thinker will expect to see another kind of matter---the only true matter in its opinion---which will no longer have anything but purely geometrical qualities, and the atoms of which will be mathematical points subject to the laws of dynamics alone.  And yet he will try to represent to himself, by an unconscious contradiction, these invisible and colourless atoms, and therefore to bring them as close as possible to ordinary matter.
    
    Then only will he be thoroughly satisfied, and he will then imagine that he has penetrated the secret of the universe.  Even if the satisfaction is fallacious, it is none the less difficult to give it up.  Thus, on opening the pages of Maxwell, a Frenchman expects to find a theoretical whole, as logical and as precise as the physical optics that is founded on the hypothesis of the ether.  He is thus preparing for himself a disappointment which I should like the reader to avoid; so I will warn him at once of what he will find and what he will not find in Maxwell.  
    
    Maxwell does not give a mechanical explanation of electricity and magnetism; he confines himself to showing that such an explanation is possible.  He shows that the phenomena of optics are only a particular case of electro-magnetic phenomena.  From the whole theory of electricity a theory of light can be immediately deduced.  
    \citep[p. 213-216]{Poincare1952}
\end{quote}


\begin{quote}
    \emph{The Mechanical Explanation of Physical Phenomena}
    
    In every physical phenomenon there is a certain number of parameters which are reached directly by experiment, and which can be measured.  I shall call them the parameters $q$.  Observation next teaches us the laws of the variations of these parameters, and these laws can be generally stated in the form of differential equations which connect together the parameters $q$ and time.  What can be done to give a mechanical interpretation to such a phenomenon?  We may endeavor to explain it, either by the movements of ordinary matter, or by those of one or more hypothetical fluids.  These fluids will be considered as formed of a very large number of isolated molecules $m$.  When may we say that we have a complete mechanical explanation of the phenomenon?  It will be, on the one hand, when we know the differential equations which are satisfied by the coordinates of these hypothetical molecules $m$, equations which must, in addition, conform to the laws of dynamics; and, on the other hand, when we know the relations which define the coordinates of the molecules $m$ as functions of the parameters $q$, attainable by experiment.  These equations, as I have said, should conform to the principles of dynamics, and, in particular, to the principle of the conservation of energy, and to that of least action.  
    
    The first of these two principles teaches us that the total energy is constant, and may be divided into two parts:
    
    (1) Kinetic energy, or \emph{vis viva}, which depends on the masses of the hypothetical molecules $m$, and on their velocities.  This I shall call $T$.  (2) The potential energy which depends only on the coordinates of these molecules, and this I shall call $U$.  It is the sum of the energies $T$ and $U$ that is constant.
    
    Now what are we taught by the principle of least action?  It teaches us that to pass from the initial position occupied at the instant $t_0$ to the final position occupied at the instant $t_1$, the system must describe such a path that in the interval of time between the instant $t_0$ and $t_1$, the mean value of the action---\emph{i.e.}, the \emph{difference} between the two energies $T$ and $U$, must be as small as possible.  The first of these two principles is, moreover, a consequence of the second.  If we know the functions $T$ and $U$, this second principle is sufficient to determine the equations of motion.  
    
    Among the paths which enable us to pass from one position to another, there is clearly one for which the mean value of the action is smaller than for all the others.  In addition, there is only [one] such path; and it follows from this, that the principle of least action is sufficient to determine the path followed, and therefore the equations of motion.  We thus obtain what are called the equations of Lagrange.  In these equations the independent variables are the coordinates of the hypothetical molecules $m$; but I now assume that we take for the variables the parameters $q$, which are directly accessible to experiment.
    
    The two parts of the energy should then be expressed as a function of the parameters $q$ and their derivatives; it is clear that it is under this form that they will appear to the experimenter.  The latter will naturally endeavor to define kinetic and potential energy by the aid of quantities he can directly observe.  If this be granted, the system will always proceed from one position to another by such a path that the mean value of the action is a minimum.  It matters little that $T$ and $U$ are now expressed by the aid of the parameters $q$ and their derivatives; it matters little that it is also by the aid of these parameters that we define the initial and final positions; the principle of least action will always remain true.
    
    Now here again, of the whole of the paths which lead from one position to another, there is one and only one for which the mean action is a minimum.  The principle of least action is therefore sufficient for the determination of the differential equations which define the variations of the parameters $q$.  The equations thus obtained are another form of Lagrange's equations.
    
    To form these equations we need not know the relations which connect the parameters $q$ with the coordinates of the hypothetical molecules, nor the masses of the molecules, nor the expression of $U$ as a function of the coordinates of these molecules.  All we need know is the expression of $U$ as a function of the parameters $q$, and that of $T$ as a function of the parameters $q$ and their derivatives---\emph{i.e.}, the expressions of the kinetic and potential energy in terms of experimental data.
    
    One of two things must now happen.  Either for a convenient choice of $T$ and $U$ the Lagrangian equations, constructed as we have indicated, will be identical with the differential equations deduced from experiment, or there will be no functions $T$ and $U$ for which this identity takes place.  In the latter case it is clear that no mechanical explanation is possible.  The \emph{necessary} condition for a mechanical explanation to be possible is therefore this: that we may choose the functions $T$ and $U$ so as to satisfy the principle of least action, and of the conservation of energy.  Besides, this condition is \emph{sufficient}.  Suppose, in fact, that we have found a function $U$ of the parameters $q$, which represents one of the parts of energy, and that the part of the energy which we represent by $T$ is a function of the parameters $q$ and their derivatives; that it is a polynomial of the second degree with respect to its derivatives, and finally that the Lagrangian equations formed by the aid of these two functions $T$ and $U$ are in conformity with the data of the experiment.  How can we deduce from this a mechanical explanation?  $U$ must be regarded as the potential energy of a system of which $T$ is the kinetic energy.  There is no difficulty as far as $U$ is concerned, but can $T$ be regarded as the \emph{vis viva} of a material system?
    
    It is easily shown that this is always possible, and in an unlimited number of ways.  I will be content with referring the reader to the pages of the preface of my \emph{\'Electricit\'e et Optique} for further details.  Thus, if the principle of least action cannot be satisfied, no mechanical explanation is possible; if it can be satisfied, there is not only one explanation, but an unlimited number, whence it follows that since there is one there must be an unlimited number.
    
    One more remark.  Among the quantities that may be reached by experiment directly we shall consider some as the coordinates of our hypothetical molecules, some will be our parameters $q$, and the rest will be regarded as dependent not only on the coordinates but on the velocities---or what comes to the same thing, we look on them as derivatives of the parameters $q$, or as combinations of these parameters and their derivatives.
    
    Here then a question occurs: among all these quantities measured experimentally which shall we choose to represent the parameters $q$? and which shall we prefer to regard as the derivatives of these parameters?  This choice remains arbitrary to a large extent, but a mechanical explanation will be possible if it is done so as to satisfy the principle of least action.
    
    Next, Maxwell asks: Can this choice and that of the two energies $T$ and $U$ be made so that electric phenomena will satisfy this principle?  Experiment shows us that the energy of an electro-magnetic field decomposes into electro-static and electro-dynamic energy.  Maxwell recognised that if we regard the former as the potential energy $U$, and the latter as the kinetic energy $T$, and that if on the other hand we take the electro-static charges of the conductiors as the parameters $q$---under these conditions, Maxwell has recognised that electric phenomena satisfies the principle of least action.  He was then certain of a mechanical explanation.  If he had expounded this theory at the beginning of his first volume, instead of relegating it to a corner of the second, it would not have escaped the attention of most readers.  If therefore a phenomenon allows of a complete mechanical explanation, it allows of an unlimited number of others, which will equally take into account all the particulars revealed by experiment.  And this is confirmed by the history of every branch of physics.  In Optics, for instance, Fresnel believed vibration to be perpendicular to the plane of polarisation; Neumann holds that it is parallel to that plane.  For a long time an \emph{experimentum crucis} was sought for, which would enable us to decide between these two theories, but in vain.  In the same way, without going out of the domain of electricity, we find that the theory of two fluids and the single fluid theory equally account in a satisfactory manner for all the laws of electro-statics.  All these facts are easily explained, thanks to the properties of the Lagrange equations.
    
    It is easy now to understand Maxwell's fundamental idea.  To demonstrate the possibility of a mechanical explanation of electricity we need not trouble to find the explanation itself; we need only know the expression of the two functions $T$ and $U$, which are the two parts of energy, and to form with these two functions Lagrange's equations, and then to compare these equations with the experimental laws.
    
    How shall we choose from all the possible explanations one in which the help of experiment will be wanting?  The day will perhaps come when physicists will no longer concern themselves with questions which are inaccessible to positive methods, and will leave them to the metaphysicians.  That day has not yet come; man does not so easily resign himself to remaining for ever ignorant of the causes of things.  Our choice cannot be therefore any longer guided by considerations in which personal appreciation plays too large a part.  There are, however, solutions which all will reject because of their fantastic nature, and others which all will prefer because of their simplicity.  As far as magnetism and electricity are concerned, Maxwell abstained from making any choice.  It is not that he has a systematic contempt for all that positive methods cannot reach, as may be seen from the time he has devoted to the kinetic theory of gases.  I may add that if in his \emph{magnum opus} he develops no complete explanation, he has attempted one in an article in the \emph{Philosophical Magazine}.  The strangeness and the complexity of the hypotheses he found himself compelled to make, led him afterwards to withdraw it.  
    
    The same spirit is found throughout his whole work.  He throws into relief the essential---\emph{i.e.}, what is common to all theories; everything that suits only a particular theory is passed over almost in silence.  The reader therefore finds himself in the presence of form nearly devoid of matter, which at first he is tempted to take as a fugitive and unassailable phantom.  But the efforts he is thus compelled to make force him to think, and eventually he sees that there is often something rather artificial in the theoretical ``aggregates'' which he once admired.
    
    \citep[p. 217-224]{Poincare1952}
\end{quote}

- ch. 13 electrodynamics, ampere vs helmholtz regarding faraday unipolar induction experiment.  a bit hard to follow

\begin{quote}
    III. \emph{Difficulties raised by these Theories.}---Helmholtz's theory is an advance on that of Amp\`ere; it is necessary, however, that every difficulty should be removed.  In both, the name ``magnetic field'' has no meaning, or, if we give it one by a more or less artificial convention, the ordinary laws so familiar to electricians no longer applly; and it is thus that the electro-motive force induced in a wire is no longer measured by the number of lines of force met by that wire.  And our objections do not proceed only from the fact that it is difficult to give up deeply rooted habits of language and thought.  There is something more.  If we do not believe in actions at a distance, electrodynamic phenomena must be explained by a modification of the medium.  And this medium is precisely what we call ``magnetic field,'' and then the electromagnetic effects should only depend on that field.  All these difficulties arise from the hypothesis of open currents.
    
    IV.  \emph{Maxwell's Theory.}---Such were the difficulties raised by the current theories, when Maxwell with the stroke of a pen caused them to vanish.  To his mind, in fact, all currents are closed currents.  Maxwell admits that if in a dielectric, the electric field happens to vary, this dielectric becomes the seat of a particular phenomenon acting on the galvanometer like a current and called the \emph{current of displacement}.  If, then, two conductors bearing positive and negative charges are placed in connection by means of a wire, during the discharge there is an open current of conduction in that wire; but there are produced at the same time in the surrounding dielectric currents of conduction.  We know that Maxwell's theory leads to the explanation of optical phenomena which would be due to extremely rapid electrical oscillations.  At that period such a conception was only a daring hypothesis which could be supported by no experiment; but after twenty years Maxwell's ideas received the confirmation of experiment.  Hertz succeeded in producing systems of electric oscillations which reproduce all the properties of light, and only differ by the length of their wave---that is to say, as violet differs from red.  In some measure he made a synthesis of light.  It might be said that Hertz has not directly proved Maxwell's fundamental idea of the action of the current of displacement on the galvanometer.  That is true in a sense.  What he has shown directly is that electromagnetic induction is not instantaneously propagated, as was supposed, but its speed is the speed of light.  Yet, to suppose there is no current of displacement, and that induction is with the speed of light; or, rather, to suppose that the currents of displacement produce inductive effects, and that the induction takes place instantaneously---\emph{comes to the same thing}.  This cannot be seen at the first glance, but it is proved by an analysis of which I must not even think of giving even a summary here.
    
    \citep[p. 238-240]{Poincare1952}
\end{quote}

- Rowland's experiment, Lorentz's theory

\begin{quote}
    According to Lorentz's theory, currents of conduction would themselves be true convection currents.  Electricity would remain indissolubly connected with certain material particles called \emph{electrons}.  The circulation of these electrons through bodies would produce voltaic currents, and what would distinguish conductors from insulators would be that the one could be traversed by these electrons, while the others would check the movement of the electrons.  Lorentz's theory is very attractive.  It gives a very simple explanation of certain phenomena, which the earlier theories---even Maxwell's in its primitive form---could only deal with in an unsatisfactory manner; for example, the aberration of light, the partial impulse of luminous waves, magnetic polarisation, and Zeeman's experiment.  
    
    A few objections still remained.  The phenomena of an electric system seemed to depend on the absolute velocity of translation of the centre of gravity of this system, which is contrary to the idea that we have of the relativity of space.  Supported by M. Cr\'emieu, M. Lippman has presented this objection in a very striking form.  Imagine two charged conductors with the same velocity of translation.  They are relatively at rest.  However, each of them being equivalent to a current of convection, they ought to attract one another, and by measuring this attraction we could measure their absolute velocity.  ``No!'' replied the partisans of Lorentz.  ``What we could measure in that way is not their absolute velocity, but their relative velocity \emph{with respect to the ether}, so that the principle of relativity is safe.''  Whatever there may be in these objections, the edifice of electrodynamics seemed, at any rate in its broad lines, definitively constructed.  Everything was presented under the most satisfactory aspect.  The theories of Amp\`ere and Helmholtz, which were made for the open currents that no longer existed, seem to have no more than purely historic interest, and the inextricable complications to which these theories led have been almost forgotten.  This quiescence has been recently disturbed by the experiments of M. Cr\'emieu, which have contradicted, or at least have seemed to contradict, the results formerly obtained by Rowland.  Numberous investigators have endeavoured to solve the question, and fresh experiments have been undertaken.  What result will they give?  I shall take care not to risk a prophecy which might be falsified between the day this book is ready for the press and the day on which it is placed before the public.  
    
    THE END
    
    \citep[p. 242-244]{Poincare1952}
    
    
\end{quote}











\section{Riezler Quotes}

Subtitle of book: Lectures of Aristotle on Modern Physics at an International Congress of Science

Riezler is channeling or pretending for argumentative and dialectic purposes to be ``Aristotle'' in his address and critiques of modern science after the turn of the century.

\begin{quote}
     The declination of your psi functions in quantum theory refers to secondary not to primary, to compound not to simple movements. The psi functions report the chances of an observation to find a place $A$ for a charge $B$, or for a charge $X$ a place $Y$. They distribute the chances of a charge to places, the chances of a place to charges. The change of chance cannot be primary motion. Chance is chance for an event. Instead of the 'ether' the probability wave does the vibrating. But what vibrates? The chance for an event.

     Again, what is an event? Here you have no answer, except the answer of classical physics. But you cannot apply this unless you decide to lift yourselves up by your own bootstraps. The situation is curious. You proceed from discovery to discovery; that I do not deny. These discoveries are firmly rooted in shifting ground. They are rules for the coincidences of numbers signifying chances for events.

     In this situation you turn to philosophy to provide you with a theory of knowledge enabling you to get around any troublesome question. It is the old dodge: the real is the observable---at least for the physicist. You first define the physicist, setting his task. Then you limit the observable: the pointer readings of possible instruments. This is the `reality' relative to your anonymous observer. I do not know whet11er this can be called theory of knowledge. Anyway it is not philosophy. It seems to me merely a definition of physics.

    But that we have already discussed. Applying this so-called theory to your new situation you argue: The only knowledge I shall and can have is knowledge of psi functions. I know all that `is' when I know all I can. It may be and usually is the case that I do not know the psi function or know it insufficiently. Then the probability this psi function reports is but relative to my insufficient knowledge. However, if I have a `maximum knowledge' of the psi function, this psi function is the condition of physics, the maximum knowledge is the thing itself.

    I understand why you assert this. You want to prevent na\"ive perceptions from interfering with the numbers of the anonymous observer. If an accurate location of an electron is not observable, then an accurate location should not be postulated. An `accurate location' of an electron is for the physicist a senseless term, to which nothing real corresponds. I approve your intention. The anonymous observer should not mix into his numbers
    concepts he cannot legitimate. Neverthe1ess, you cannot stop here. Probabilities are probabilities of something. A probability of nothing is still more senseless than an accurate location of an electron.

    The physicists of past times have developed your preconceptions of tho measured and measurable quantities on the strength of occurrences on a large scale---without any insight into the microworld. Now you find that these quantities are measurable with but limited accuracy. Perhaps they are not really the right concepts. If they are not, then the task would be to discover the right concepts, the true building stones of Being. But this you do not attempt. Numbers, psi functions, are observable and therefore real, even though they are probabilities of nothing. No! my friends.

    \citep[p. 31-32]{Riezler1940}
\end{quote}