\section{Riezler Quotes}

Subtitle of book: Lectures of Aristotle on Modern Physics at an International Congress of Science

Riezler is channeling or pretending for argumentative and dialectic purposes to be ``Aristotle'' in his address and critiques of modern science after the turn of the century.

\begin{quote}
     The declination of your psi functions in quantum theory refers to secondary not to primary, to compound not to simple movements. The psi functions report the chances of an observation to find a place $A$ for a charge $B$, or for a charge $X$ a place $Y$. They distribute the chances of a charge to places, the chances of a place to charges. The change of chance cannot be primary motion. Chance is chance for an event. Instead of the 'ether' the probability wave does the vibrating. But what vibrates? The chance for an event.

     Again, what is an event? Here you have no answer, except the answer of classical physics. But you cannot apply this unless you decide to lift yourselves up by your own bootstraps. The situation is curious. You proceed from discovery to discovery; that I do not deny. These discoveries are firmly rooted in shifting ground. They are rules for the coincidences of numbers signifying chances for events.

     In this situation you turn to philosophy to provide you with a theory of knowledge enabling you to get around any troublesome question. It is the old dodge: the real is the observable---at least for the physicist. You first define the physicist, setting his task. Then you limit the observable: the pointer readings of possible instruments. This is the `reality' relative to your anonymous observer. I do not know whet11er this can be called theory of knowledge. Anyway it is not philosophy. It seems to me merely a definition of physics.

    But that we have already discussed. Applying this so-called theory to your new situation you argue: The only knowledge I shall and can have is knowledge of psi functions. I know all that `is' when I know all I can. It may be and usually is the case that I do not know the psi function or know it insufficiently. Then the probability this psi function reports is but relative to my insufficient knowledge. However, if I have a `maximum knowledge' of the psi function, this psi function is the condition of physics, the maximum knowledge is the thing itself.

    I understand why you assert this. You want to prevent na\"ive perceptions from interfering with the numbers of the anonymous observer. If an accurate location of an electron is not observable, then an accurate location should not be postulated. An `accurate location' of an electron is for the physicist a senseless term, to which nothing real corresponds. I approve your intention. The anonymous observer should not mix into his numbers
    concepts he cannot legitimate. Neverthe1ess, you cannot stop here. Probabilities are probabilities of something. A probability of nothing is still more senseless than an accurate location of an electron.

    The physicists of past times have developed your preconceptions of tho measured and measurable quantities on the strength of occurrences on a large scale---without any insight into the microworld. Now you find that these quantities are measurable with but limited accuracy. Perhaps they are not really the right concepts. If they are not, then the task would be to discover the right concepts, the true building stones of Being. But this you do not attempt. Numbers, psi functions, are observable and therefore real, even though they are probabilities of nothing. No! my friends.

    \citep[p. 31-32]{Riezler1940}
\end{quote}