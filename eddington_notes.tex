%\tableofcontents

\section{Eddington's \emph{The Philosophy of Physical Science}}

\subsection{Chapter I: Scientific Epistemology}

- scientific epistemology more involved and inseperable in revolutions compared to past where 

\begin{quote}
    To advance science and to philosophise on science were essentially distinct activities. In the new movement scientific epistemology is much more intimately associated with science.  For developing the modern theories of matter and radiation a definite epistemological outlook has become a necessity; and it is the direct source of the most far-reaching scientific advances.
    
    We have discovered that \emph{it is actually an aid in the search for knowledge to understand the nature of the knowledge which we seek.}
    
    By making practical application of our epistemological conclusions we subject them to the same kind of observational control as physical hypotheses.  If our epistemology is at fault, it will lead to an \emph{impasse} in the scientific developments proceeding from it; that warns us that our philosophical insight has not been deep enough, and we must cast about to find what has been overlooked.  In this way scientific advances which result from epistemological insight have in turn educated our epistemological insight.  Between science and scientific epistemology there has been a give and take by which both have greatly benefited.
    
    In the view of scientists at least, this observational control gives to modern scientific epistemology a security which philosophy has not usually been able to attain.  It introduces also the same kind of progressive development which is characteristic of science, but not higherto of philosophy.  We are not making a series of shots at ultimate truth, which may hit or miss.  What we claim for the present system of scientific philosophy is that it is an advance on that which went before, and that it is a foundation for the advances which will come after it.
    
    \citep[p. 5]{Eddington1939}
\end{quote}

- seems to be a nice connection to Bell's theorem and recent Nobel prize experimental work?

- positivism as scientific epistemology for physics specifically?  counterfactuals, theory laden observation, 

\begin{quote}
    For the truth of the conclusions of physical science, observation is the supreme Court of Appeal.  It does not follow that every item which we confidently accept as physical knowledge has actually been certified by the Court; our confidence is that it would be certified by the Court if it were submitted.  But it does follow that every item of physical knowledge is of a form which might be submitted to the Court.  It must be such that we can specify (although it may be impracticable to carry out) an observational procedure which would decide whether it is true or not.  Clearly a statement cannot be tested by observation unless it is an assertion about the results of observation.  \emph{Every item of physical knowledge must therefore be an assertion of what has been or would be the result of carrying out a specified observational procedure.}
    
    I do not think that anyone---least of all, those who are critical of the modern tendencies of physics---will disagree with the first axiom of scientific epistemology, namely that the knowledge obtained by the methods of physical science is limited to observational knowledge in the sense explained above.  We do not deny that knowledge which is not of an observational nature may exist, e.g. the theory of numbers in pure mathematics; and non-committally we may allow the possibility of other forms of insight of the human mind into a world outside itself.  But such knowledge is beyond the borders of physical science, and therefore does not enter into the description of the world introduced in the formulation of physical knowledge.  To a wider synthesis of knowledge, of which physical knowledge is only a part, we may perhaps correlate a ``world'' of which the physical universe is only a partial aspect.  But at this stage of our inquiry we limit the discussion to physical knowledge, and therefore to a physical universe from which, by definition, all characteristics which are not the subject of physical knowledge are excluded.  
    
    A distinction is commonly made between observational and theoretical knowledge; but in practice the terms are used so loosely as to deprive the classification of all real significance.  The whole development of physical science has been a process of combining theory and observation; and in general every item of physical knowledge---or at least every item to which attention is ordinarily directed---has a partly observational and partly theoretical basis. 
    
    \citep[p. 9-10]{Eddington1939}
\end{quote}

\begin{quote}
    Thus our axiom that all physical knowledge is of an observational nature is not to be understood as excluding theoretical knowledge.  I \emph{know} the position of Jupiter last night.  That is knowledge of an observational nature; it is possible to detail the observational procedure which yields the quantities (right ascension and declination) which express my knowledge of the planet's position.  As a matter of fact I did not follow this procedure, nor did I learn the position from anyone who had followed the procedure; I looked it up in the Nautical Almanac.  That gave me the result of a computation according to planetary theory.  Present-day physics accepts that theory and all its consequences; that is to say, it admits the calculated position as a foreknowledge of the results which would be obtained by carrying out the recognised observational procedure.  Of my two pieces of knowledge, namely knowledge of the results of a mathematical computation and foreknowledge of the results of an observational procedure, it is the latter which I assert when I claim to know the position of Jupiter.  If, on submission to the Court of Appeal, my foreknowledge of the result of the observational procedure proves to be incorrect, I shall have to admit that I was mistaken and did not know the position of Jupiter; it will be no use my urging that my knowledge of the result of the mathematical computation was correct.
    
    It is the essence of acceptance of a theory that we agree to obliterate the distinction between knowledge derived from it and knowledge derived from actual observation.  It may seem one-sided that the obliteration of the distinction should render all physical knowledge observational in nature.  But not even the most extreme worshipper of theory has proposed the reverse---that in accepting the results of an observational research as trustworthy we elevate them to the status of theoretical conclusions.  The one-sidedness is due to our acceptance of observation, not theory, as the supreme Court of Appeal.
    
    \citep[p. 10-11]{Eddington1939}
\end{quote}

- counterfactual hypothetico-observational physical knowledge ``means knowledge of the result of a hypothetical observation, not hypothetical interpretation of the result of an actual observation.'' -p. 12 footnote

\begin{quote}
    Although it would be true to assert that 240,000 miles is the result of an actual observational procedure of swinging pendulums, etc., that is not what we intend to assert when we say that the distance of the moon is 240,000 miles.  By employing accepted theory we have been able to substitute for the actual observational procedure a hypothetical observational procedure which would yield the same result if it were carried out.  The gain is that hypothetico-observational knowledge can be systematised and gathered into a coherent whole, whereas actual observational knowledge is sporadic and desultory.
    
    One cannot help feeling a misgiving that hypothetico-observational knowledge is not entirely satisfactory from a logical standpoint.  What exactly is the status of conditional knowledge if the condition is not fulfilled?  Can any sense at all be attributed to a statement that if something, which we know did not happen, had happened, then certain other things would have happened?  Yet I cannot help prizing my knowledge that 240,000 x 1760 yard-sticks \emph{would} reach from here to the moon, although there is no prospect that they ever \emph{will} do so.
    
    \citep[p. 13]{Eddington1939}
\end{quote}

- generalization of hypothetico-observational knowledge into laws vs systematisation

\subsection{Chapter II: Selective Subjectivism}

-ch 2, fish net ichthyologist analogy w/ two generalizations: all fish are greater than 2 inches in size, all fish have gills.

- generalizing data vs generalizing the epistemological and instrumental methods used to generate/gather data

- distinction between `scientific philosophy' (logical empiricism?) and what is done in practice, where epistemic generalisations are more secure

\begin{quote}
    If the ichthyologist extends his investigations, making further catches, perhaps in different waters, he may any day bring up a sea-creature without gills and upset his second generalisation.  If this happens, he will naturally begin to distrust the security of his first generalisation.  His fear is needless; for the net can never bring up anything that it is not adapted to catch.
    
    \emph{Generalisations that can be reached epistemologically have a security which is denied to those that can only be reached empirically.}
    
    It has been customary in scientific philosophy to insist that the laws of nature have no compulsory character; they are uniformities which have been found to occur hitherto in our limited experience, but we have no right to assert that they will occur invariably and universally.  This was a very proper philosophy to adopt as regards empirical generalisations---it being understood, of course, that no one would be so foolish as to apply the philosophy in practice.  
    
    \citep[p. 19]{Eddington1939}
\end{quote}


\begin{quote}
    The situation is changed when we recognise that some laws of nature may have an epistemological origin.  These are compulsory; and when their epistemological origin is established, we have a right to our expectation that they will be obeyed invariably and universally.  The process of observing, of which they are a consequence, is independent of time or place.
    
    But, it may be objected, can we be sure that the process of observing\footnote{The standard specification of the procedure of observing must be sufficiently detailed to secure a unique result of the observation.  It is the duty of the observer to secure that all attendant circumstances which can affect the result, e.g., temperature, absence of magnetic field, etc., are in accordance with specification.  Epistemological laws governing the results of the observation are such as are inferable solely from the fact that the procedure was as specified.  The contingency referred to in this paragraph is exemplified by the fact that it is impossible to make a really ``good'' observation of length in a strong magnetic field, because the standard specification of the procedure of determining length requires us to eliminate magnetic fields (p. 80).} is unaffected by time or place?  Strictly speaking, no.  But if it is affected---if position in time and space or any other circumstance prevents the observational procedure from being carried out precisely according to the recognised specification---we can (and do) call the resulting observation a ``bad observation''.  Those who resent the idea of compulsion in scientific law may perhaps be mollified by the concession that, although it can no longer be accepted as a principle of scientific philosophy that the laws of nature are uncompulsory, there is no compulsion that our actual observations shall satisfy them, for (unfortunately) there is no compulsion that our observations shall be \emph{good} observations.
    
    What about the remaining laws of nature, not of an epistemological origin, and therefore, so far as we know, non-compulsory?  Must they continue to mar the scheme as a source of indefensible expectations, which nevertheless are found to be fulfilled in practice?  Before worrying about them, it will be well to wait till we see what is left of the system of natural law after the part which can be accounted for epistemologically has been removed.  There may not be anything left to worry about.  
    
    The introduction of epistemological analysis in modern physical theory has not only been a powerful source of scientific progress, but has given a new kind of security to its conclusions.  Or, I should rather say, it has put a new kind of security within reach.  
    
    \citep[p. 20-21]{Eddington1939}
\end{quote}



- epistemologist observes observers and helps regulate by figuring out what scientific observers actually observe and what we can say about *good* observations

\begin{quote}
    The epistemologist accordingly does not study the observers as organisms whose activities must be ascertained empirically in the same way that a naturalist studies the habits of animals.  He has to pick out the good observers---those whose activities follow a conventional plan of procedure.  What the epistemologist must get at is this plan.  Without it, he does not know which observers to study and which to ignore; with it, he need not actually watch the good observers who, he knows already, are merely following its instructions, since otherwise they would not be good.  
    
    The plan must be sought for in the mind of the observer, or in the minds of those from whom he has derived his instructions.  The epistemologist is an observer only in the sense that he observes what is in the mind.  But that is a pedantic description of the way in which we discover a plan conceived in anyone's mind.  We learn the observer's plan by listening to his own account of it and cross-questioning him.
    
    \citep[p. 23]{Eddington1939}
\end{quote}

- epistemological knowledge is a priori in a legitimate sense due to the method of analysis, compared to the knowledge of observations which is a posteriori.  doesn't mean no objective knowledge (of objective world), but there is a selective and subjective aspect which is more certain (and recognition of this is called by Eddington \emph{selective subjectivism}, and mathematical structures can characterize this which is why he also considered calling his position \emph{structuralism})

\begin{quote}
    It seems appropriate to call the philosophical outlook that we have here reached \emph{selective subjectivism}.  ``Selective'' is to be interpreted broadly.  I do not wish to assert that the influence of the procedure of observing on the knowledge obtained is confined to simple selection, like passing through a net.  But the term will serve to remind us that the subjective and the objective can be combined in other ways than by mere addition.  In mathematics a very general type of such combination is that of operator and operand, selective operators being a particular case.
    
    Selection implies something to select from.  It seems permissible to conclude that the material on which the selection is performed is objective.  The only way to satisfy ourselves of this is to examine carefully the ways in which subjectivity can creep into physical knowledge through the procedure of observing.  So far as I can see, selection or operations mathematically akin to it cover the whole range of possibility; that is to say, the whole subjectivity is comprised in operations of a selective type.  The subjectivity being confined to the operators, the ultimate operand must be free from subjectivity.  
    
    I see no reason to doubt the foregoing argument, but it depends on a vigilance of scrutiny which I cannot guarantee as conclusive.  ``Objective'' is essentially a negative characteristic (non-subjective) of knowledge, although we regard it as a positive characteristic of the thing to which the knowledge refers; and it is always more difficult to demonstrate a negative than a positive conclusion.  I accept an objective element in phyiscal knowledge on, I think, reasonably strong grounds, but not with the same assurance as the subjective element which is easily demonstrable.  
    
    Selective subjectivism, which is the modern scientific philosophy, has little affinity with Berkeleian subjectivism, which, if I understand rightly, denies all objectivity to the external world.  In our view the physical universe is neither wholly subjective nor wholly objective---nor a simple mixture of subjective and objective entities or attributes.
    
    \citep[p. 26-27]{Eddington1939}
\end{quote}

- he distinguishes from Berkeley, and also indicates that he thinks he is \emph{reconstructing} modern scientific philosophy/epistemology, not necessarily to prescribe but describing to a certain degree.

\subsection{Chapter III: Unobservables}
- ch 3 Unobservables.  micro vs molar physics.

- formulating a principle of (general) relativity

\begin{quote}
    Perhaps the nearest approach to a formulation of the principle is the statement that we observe only \emph{relations} between physical entities.  \citep[p. 31]{Eddington1939}
\end{quote}


- epistemologist concerned with ``What is it we really observe?''

\begin{quote}
    Dividing physicists into three classes---relativity physicists, quantum physicists, experimental physicists---the relativity physicist studies the hard facts of observation.  The quantum  physicist follows the same principle as far as he can; but owing to the more intricate and more remote nature of his subject, the aim of constructing a theory which shall embrace only the observable facts represents his ideal rather than his achievement.  As for the experimental physicist, I will only say that because a man works in a laboratory it does not follow that he is not an incorrigible metaphysician.
    
    \citep[p. 32-33]{Eddington1939}
\end{quote}


\begin{quote}
    Since aether is not matter, it cannot be assumed \emph{a priori} that the usual attributes of matter---density, rigidity, momentum, etc.---are also attributes of the aether.  Accordingly the hypothesis to be tested is that velocity, although a well-known attribute of matter, is not one of the attributes of the aether.  Put in this way, it is not a truth that could have been foreseen \emph{a priori}; it is a mildly surprising, but clearly possible, conclusion deduced \emph{a posteriori} from the null result of experiments designed to detect effects which would be expected if there existed a luminiferous aether with the type of structure to which velocity could be attributed.
    
    This attitude is popular with those who dislike the epistemological inquiry associated with the new developments of physics.  It is so easy to cut short an argument one does not want to understand by saying: ``I am not interested in your reasons, but I am quite willing to try any conclusion you may have reached as a hypothesis to be tested by observation.  Then, if it is confirmed, it will take rank with the other confirmed hypotheses of physics, and we shall not need your arguments.''  By this kind of short-circuiting, the more difficult considerations are cut out of the subject; and we can embark at once on the straightforward mathematical deduction of the consequences of the hypothesis with a view to observational test.  Thus the new wine is put into the old bottles.  It does not burst the bottles; but it loses most of its invigorating---my opponents would perhaps say, its intoxicating---qualities.  
    
    \citep[p. 34-35]{Eddington1939}
\end{quote}

- ``unobservability is a matter of epistemological principle, not of physical hypothesis''

\begin{quote}

    As a further example, it was pointed out ten years ago that when we are dealing with particles, such as electrons, which are indistinguishable from one another observationally, the ordinary coordinate $\xi = x_2 - x_1$ of one particle relative to another is not an observable; the observable in this case is a type of quantity previously unfamiliar in analysis, namely a ``signless coordinate'' $\eta = \pm \xi$.  Up to the present, quantum physicists have chosen to ignore this imposture; and the modern text-books still adhere to the erroneous theory of a system of two such particles, which assumes the observable to be $\xi$.  They have thereby missed the opening for a much needed advance.
    
    I have mentioned this last example because it is a clear case in which unobservability is a matter of epistemological principle, not of physical hypothesis.  For simplicity, consider particles in one dimension only, say east and west.  If we have a green ball and a red ball, we can observe that the green ball is, say, 5 inches west of the red ball.  Accordingly, for purposes of description, we introduce an observable quantity $\xi$ which states the distance of the green ball from the red ball measured towards the west; a negative value of $\xi$ will indicate that the green ball is to the east.  But suppose instead that we have two balls exactly alike in colour, and with no distinction at all that we can observe.  In such a system there is no observable corresponding to $\xi$.  We can observe that the balls are 5 inches apart in the east-west line, and we can introduce an observable $\eta$ which states the distance apart.  But, unlike $\xi$, $\eta$ is a signless quantity.
    
    It is a natural mistake to apply the ordinary theory of the observable behavior of particles (\emph{particle mechanics}, as we call it) to protons and electrons, overlooking that at an early stage in that theory, namely in introducing and defining a relative coordinate $\xi$, it was taken for granted that the particles could be distinguished observationally.  This mechanics becomes inapplicable when $\xi$ is unobservable.  For protons and electrons we have a modified mechanics with $\eta$ as the observable.  This fundamental difference in the mechanics must be followed up mathematically; and although the problem is rather difficult, I think it is rigorously deducible that the difference is equivalent to a force between the particles which is actually the well-known Coulomb force.  That is to say, the electrostatic (Coulomb) force between electrons and protons is not an ``extra'' arising we know not why, but is simply a term which had dropped out in the ordinary derivation of the equations through the oversight of taking $\xi$ instead of $\eta$ as the observable, and had therefore to be re-inserted empirically.
    
    Those unfamiliar with wave-mechanics may be astonished that there should be a difference between the mechanics of distinguishable particles and the mechanis of indistinguishable particles.  But it ought not to surprise quantum physicists, since it is universally admitted that there is a difference in their statistics, which is no less mysterious.  Indeed I have never been able to understand why those who are well aware of the important consequences of indistinguishability in large assemblies do not trouble to examine its precise consequences in smaller systems.  Whether we consider the well-known effect on the statistics of large assemblies or the less well-known effect on the mechanics of a system of two particles, the conclusions appear incredible unless we bear in mind the subjectivity of the world described by physics and of all that it is said to contain.  It is naturally objected that the particles cannot be affected by our inability to distinguish them, and it is absurd to suppose that they modify their behaviour on that account.  That would be true if we were referring to wholly objective particles and wholly objective behaviour.  But our generalizations about their behaviour---the laws of mechanics---describe properties imposed by our procedure of observation, as the generalisations about catchable fish were imposed by the structure of the net.  The objective particles are unconcerned with our inability to distinguish them; but they are equally unconcerned with the behaviour which we attribute to them partly as a consequence of our failure to distinguish them.  It is this observable behaviour, and not the objective behaviour, that \emph{we} are concerned with. 
    
    \citep[p. 35-37]{Eddington1939}
\end{quote}

- progress/advance as reduction in number of fundamental hypotheses.  three ways: abandonment of mechanical ideal of explanation (math instead of hyp. stuff), unification (e.g. light/optics with electromagnetism), and replacement by epistemological principles (exemplified by relativity)

\begin{quote}
    With the coming of relativity theory yet a third method of reducing the number of hypotheses crept in, namely the replacement of physical hypotheses by epistemological principles.  We have already noticed the way in which an epistemological conclusion can play the same part as a physical hypothesis so far as observational consequences are concerned.  
    
    We have seen (p. 20) that laws and properties which have an epistemological origin are compulsory and universal.  It may be added that, in some cases at least, they are exact.  For the unobservability of certain quantities---which is the most common form of statement of an epistemological principle---is traced to a logical contradiction in their definitions; and the consequences (in so far as they are reached by logical deduction alone, and not by combination with more or less uncertain and inexact hypotheses) are quite definite.  The pervasion of fundamental physics by epistemology has therefore greatly changed its character, and brought exactitude within reach.  So long as the methods were wholly \emph{a posteriori}, there was no warrant for regarding the deduced laws of nature as better than approximations.  
    
    To avoid misapprehension it is best to state here (prematurely) that although we now recognise laws which we can confidently assert are exact, the subject-matter of these exact laws is probability.  There is therefore not a corresponding precision in the laws of observational phenomena (as distinguished from the laws of \emph{probability} of the phenomena); and, notwithstanding its newly acquired exactness, the system of fundamental physical laws is indeterministic.
    
    \citep[p. 45-46]{Eddington1939}
\end{quote}

- I think that he probably doesn't mean irreducibly indeterministic, only that the character of the laws as we represent in our minds for now/then says that that is the case.


\subsection{Chapter IV: The Scope of Epistemological Method}


- ch 4 The Scope of Epistemological Method

\begin{quote}
    According to the classical conception of microscopic physics, our task was to discover a system of equations which connects the positions, motions, etc. of the particles at one instant, with the positions, motions, etc. at a later instant.  This problem has proved altogether baffling; we have no reason to believe that any determinate solution exists, and the search has been frankly abandoned.  Modern quantum theory has substituted another task, namely to discover the equations which connect knowledge of the positions, motions, etc. at one instant with the knowledge of the positions, motions, etc. at a later instant.  The solution of this problem appears to be well within our power.
    
    The mathematical symbolism describes our knowledge, and the mathematical equations trace the change of this knowledge with time.  Our knowledge of physical quantities is always more or less inexact; but the theory of probability enables us to give an exact specification of inexact knowledge, including a specification of its inexactitude.  The introduction of probability into physical theories emphasises the fact that it is knowledge that is being treated.  For probability is an attribute of our knowledge of an event; it does not belong to the event itself, which must certainly occur or not occur.  
    
    Wave mechanics investigates the way in which probability redistributes itself as time elapses; it analyses it into waves and determines the laws of propagation of those waves.  Generally the waves tend to diffuse; that is to say, our knowledge of the position (or of any other characteristic) of a system becomes vaguer the longer the time elapsed since an observation was made.  A sudden accession to knowledge---our becoming aware of the result of a new observation---is a discontinuity in the ``world'' of probability-waves; the probability is reconcentrated, and the propagation starts again from the new distribution.  There are exceptional forms of probability distribution of certain of the attributes of microscopic systems which do not diffuse, or diffuse very slowly; so that our knowledge of these attributes does not so rapidly grow out of date.  Particular attention is lavished on these ``steady states'' and on the equations determining them, since they provide a basis for long-range predictions.
    
    The statement often made, that in modern theory the electron is not a particle but a wave, is misleading.  The ``wave'' represents our knowledge of the electron.  The statement is, however, an inexact way of emphasising that the knowledge, not the entity itself, is the direct object of our study; and it may perhaps be excused by the fact that the terminology of quantum theory is now in such utter confusion that it is well-nigh impossible to make clear statements in it.  The term ``electron'' has at least three different meanings\footnote{Namely, the particle represented by a Dirac wave-function, the particle introduced in second quantisation and the particle represented by the internal (relative) wave-function of a hydrogen atom.} in common use in quantum theory, in addition to its loose application to the probability wave itself.
    
    Wave mechanics shows us immediately why the distinction between observatables and unobservables is so essential.  A ``good'' observation of a quantity, although it does not determine the quantity precisely, narrows down the range in which it is likely to lie.  It creates a condensation in the probability distribution of the quantity or, as we usually say, forms a wave packet in it.  The method of wave mechanics is to investigate the wave equations which govern the propagation of waves from such a source.  But if the quantity is unobservable, these wave packets cannot be formed.  A study of the propagation of waves which there is no means of producing can have no application to physics; and a theory which professes to deduce observationally verifiable results by such analysis is evidently vitiated by a mis-identification.
    
    \citep[p. 50-52]{Eddington1939}
\end{quote}

- absolutely wild, very clear notion of epistemic wave function and arguably a form of qbism from 1939 when this was written

- reception of Einstein's theory, theory is not identical to math, what the role of a physicist actually is (not just applying math)

\begin{quote}
    When Einstein's theory arrived, which not only propounded a new epistemology but applied it to determine the law of gravitation and other practical consequences, physicists were puzzled how to classify it.  Some argued that it was philosophy, \emph{alias} metaphysics, and must be rejected out of hand.  Others conceded that the formulae appeared to agree with observation and accomplished a valuable systematisation of knowledge, but believed that a ``genuinely physical'' interpretation of its meaning would in time supplant the epistemological jargon which at present envelops it.  Fewer realised that the new epistemological outlook is the very heart of the theory, supplanting a fallacious system of thought which was barring progress.  Even now we often find authors, who are by no means ignorant of the reasons for the change of thought, propounding theories for which they claim the advantage that they involve only Newtonian conceptions.  As though it could be an advantage to incorporate a fallacious and obsolete view of the nature of observational knowledge!
    
    This vagueness and inconsistency of the attitude of most physicists is largely due to a tendency to treat the mathematical development of a theory as the only part which deserves serious attention.  But in physics everything depends on the insight with which the ideas are handled before they reach the mathematical stage.
    
    The consequence of this tendency is that a theory is very commonly identified with its leading mathematical formulae.  We continually find special relativity theory identified with the Lorentz transformation, general relativity with the transformation to generalised coordinates, quantum theory with the wave equation or the commutation relations.  It cannot be too strongly urged that neither relativity theory nor quantum theory are summed up in fool-proof formulae for use on all occasions.  A relativist is not a man who employs Lorentz-invariant formulae (which were introduced some years before the relativity theory appeared), but one who understands in what circumstances formulae ought to have Lorentz-invariance; nor is he a man who transforms equations into generalised coordinates (a practice at least a century old), but one who understands in what circumstances a special system of coordinates would be inapplicable.  In quantum problems allowance must be made for the backward state of the theory; and the world is still awaiting a quantist who understands in what circumstances the standard wave equation and the commutation relations are applicable---as distinct from one who merely applies them and hopes for the best.
    
    \citep[p. 55-56]{Eddington1939}
\end{quote}

- Eddington sees little reason, given relativity's clear dependence/import as being subjective/epistemological, to suppose that there are parts of fundamental theory which are subjective and parts that are objective... nor a reason to try when it seems probably better to just go full blown idealism.  Pragmatic reasons?  SLIDE DOWN IDEALIST GRADIENT

\begin{quote}
    I do not see how anyone who accepts the theory of relativity can dispute that there has been some replacement of physical hypotheses by epistemological principles; nor do I think that those who accept the theory with understanding will be inclined to dispute it.  The more controversial question is, How far can this replacement extend?  Here my conclusion, based on purely scientific investigation, is much more drastic than that of most of my colleagues.  I believe that the whole system of fundamental hypotheses can be replaced by epistemological principles.  Or, to put it equivalently, all the laws of nature that are usually classed as fundamental can be foreseen wholly from epistemological considerations.  They correspond to \emph{a priori} knowledge, and are therefore \emph{wholly subjective}.
    
    I am sorry to have to put in the forefront what will generally be regarded as an individual scientific conclusion; but this cannot be avoided.  I think I can see a clear philosophy emerging from the conclusion that the system of fundamental laws is wholly subjective.  I cannot see any coherent philosophy emerging from the conclusion that some are subjective and some objective.  Immediately I start on that line I am beset with objections and perplexities which I do not know how to meet.  I do not condemn it on that account; perhaps with a great deal more thought a way of progress could be seen.  But there is no inducement to spend my time trying to overcome the difficulties of a philosophy associated with scientific beliefs which I do not share.  No one can contemplate entering on a difficult research based on premises which he has reason to believe erroneous.  You will find plenty of philosophies of objective natural law; you will find here a philosophy of subjective natural law.  If ever a philosophy of mixed subjective-objective natural law is developed, it will not be by me, for I am convinced that there is no scientific support for such a philosophy.
    
    \citep[p. 56-57]{Eddington1939}
\end{quote}

\begin{quote}
    My conclusion is that not only the laws of nature but the constants of nature can be deduced from epistemological considerations, so that we can have \emph{a priori} knowledge of them.
    
    Treating the scheme of natural law as a whole, as it is set out in the fundamental equations of physics, four constants of nature which are pure numbers\footnote{Formed by eliminating our three arbitrary units (centimetre, gram and second) from the seven constants of nature ordinarily recognised. (\emph{New Pathways in Science}, p. 232)} are involved.  These I find to be predictable \emph{a priori}.  I focus attention on these, because it is a more stringent test of the power of the epistemological method to provide a number (verifiable in some cases to about 1 part in 1000) than to provide forms of law.  I think that the classical physicist had an inner feeling that the inverse square law was a natural form of weakening of an effect by distance, which might be expected \emph{a priori} to apply to gravitation---though it would, of course, be contrary to his principles to acknowledge any \emph{a priori} expectation.  
    
    \citep[p. 58-59]{Eddington1939}
\end{quote}

- one constant in particular: cosmical number, i.e. number of particles in the universe (which in a later chapter he gives an exact number...!)

- Eddington openly struggling with probability in subjective view of natural laws... perhaps a Bayesian view would merge well?!  [WORK ON]

\begin{quote}
    The ``law of chance'' is not usually counted as a fundamental law of physics, and I do not include it among the laws that can be foreseen wholly from epistemological considerations.  But according to the modern system of physics all our predictions of phenomena are predictions of what will probably happen, and are based on an assumption of non-correlation of the behaviour of individual particles which is derived from the law of chance.  \emph{Without an appeal to the law of chance physics is unable to make any prediction of the future.}  The law of chance might therefore be claimed to be the most fundamental and indispensable of all physical laws.  The reason why it is omitted is that, from the ordinary point of view, randomness is a negation of law; and it seems unnecessary to lay down a law saying that there is no law.  But the ordinary view takes it for granted that the physical universe, and the particles into which we analyse it, are wholly objective; and the status of the law of chance (or non-correlation) requires reconsideration when applied to a partly subjective universe.  It is impossible to treat this point fully until a late stage of the discussion.  The view finally adopted will be found on pp. 180, 218.  If in the meantime the reader finds my argument tending apparently to a more and more incredible conclusion, he may await a later twist that will soften it into something which will, I think, not too grossly affront his commonsense.
    
    \citep[p. 61]{Eddington1939}
\end{quote}

\begin{quote}
    Without further apology I shall now assume the reader's assent to the proposition that all the fundamental laws and constants of physics can be deduced unambiguously from \emph{a priori} considerations, and are therefore wholly subjective.  
    
    \citep[p. 62]{Eddington1939}
\end{quote}

- shifts burden of proof to anyone who thinks the laws are objective, fishing net analogy again

\begin{quote}
    But if our purpose is to determine laws of objective origin coming through to us in a form modified by subjective selection, I do not think the best way is to suppress all theories about the objective world---at any rate as working hypotheses.  But at first sight the progress of physics seems to contradict this; for it was just when hypothesis about the objective world was abandoned, and we turned to a direct study of physical knowledge, that progress became astonishingly rapid.
    
    The explanation is simple.  All this progress relates to subjective law.  It all relates to uniformities imposed on the results of observation by the procedure of observation.  
    
    \citep[p. 62]{Eddington1939}
\end{quote}

- laws of nature (dynamical/differrential equations) vs special facts (initial conditions), Laplace determinism in classical physics, not applicable in new physics.  New physics, being epistemological, does not demarcate the same way between these, and what we might think as a special fact he claims to show is also epistemological (``subjective'').

\begin{quote}
    But this mode of distinction is possible only in a deterministic universe.  In the current indeterministic system of physics, there is no corresponding demarcation between the laws and the special facts of nature.  The present system of fundamental laws does not furnish a complete set of rules for the calculation of the future.  It is not even part of such a set, for it is concerned only with the calculation of probabilities; and if ever the search for a scheme of definite prediction is renewed, it will be necessary to start again from the beginning on different lines.  The part played by the special facts is also altered.  The special facts, which distinguish the actual universe from all other possible universes obeying the same laws, are not given once for all at some past epoch, but are being born continually as the universe follows its unpredictable course.  Moreover, in the differential equations of quantum theory the boundary conditions are not the objective facts but the knowledge we happen to possess about them.
    
    The simple demarcation in classical theory between fundamental laws of nature and special facts is associated with determinism and cannot be carried over into the modern theory.  But, approaching the question from the point of view of subjectivity, a new line of demarcation appears.  We have found that the supposedly fundamental laws are wholly subjective.  It is only reasonable that the part of our knowledge which is wholly subjective should be of a recognisably different type from that which involves the objective characteristics of the universe.  It appears that this difference was not overlooked by the earlier physicists; and we find the region to be annexed to pure subjectivity already marked out under another name, viz. ``fundamental''.
    
    The special facts, on the other hand, cannot be inferred from epistemological considerations and are not wholly subjective.  It is the essence of our conception of a special fact that it might quite well have been otherwise---that there is no \emph{a priori} reason why it should be what it is.  It is true that many have held the view that the laws of nature might quite well have been otherwise; but they would scarcely assert that this is an inseperable part of the conception of a law of nature.  Everyone recognises that it is in some sense taking a greater liberty with the universe to imagine the laws of nature to have been different than to imagine the special facts to have been different.
    
    Results deducible by the \emph{a priori} epistemological method are compulsory, and it is therefore impossible that the method should be extended to predict the special facts, which ``might quite well have been otherwise''.  I am afraid that before I finish I shall have persuaded the docile reader to believe so many ``impossible'' things that the world will make little impression on him, and he will not jib at impossibility when I want him to.  Let me then put the point rather differently.  If by an advance of epistemological theory we succeed in predicting one of the so-called special facts in a wholly \emph{a priori} way, we shall at once amend the classification: ``Clearly we were mistaken in supposing that it was a special fact.  Now that we see more clearly into its origin, we realise that there is a law of nature which compels it to be so.''
    
    The cosmical number affords a good example of such a change of view.  Regarded as the number of particles in the universe, it has generally been looked upon as a special fact.  A universe, it is held, could be made with any number of particles; and, so far as physics is concerned, we must just accept the number allotted to our universe as an accident or as a whim of the Creator.  But the epistemological investigation changes our idea of its nature.  A universe cannot be made with a different number of elementary particles---consistently with the scheme of definitions by which the ``number of particles'' is assigned to a system in wave-mechanics.  We must therefore no longer look on it as a special fact about the universe, but as a parameter occurring in the laws of nature, and, as such, part of the laws of nature.
    
    \citep[p. 63-65]{Eddington1939}
\end{quote}

- special facts somehow changing and evidenced by uncertainty principle

\begin{quote}
    Within the limits of the uncertainty principle they are ever-changing as the moments pass by.
    
    The special facts are partly subjective and partly objective, depending partly on our procedure in obtaining observational knowledge and partly on what there is to observe.  To separate the subjective or objective elements completely, we must consider laws; since a law or regularity may originate wholly in our procedure of observation or wholly in the objective world.  It may be questioned whether we could ever isolate an objective law as completely as a subjective law, since it would have to be presented to us \emph{via} our subjective forms of thought; but at least we could detect a regularity and recognise that is origin was objective, even if we could only describe it in subjective terms.
    
    
    
    \citep[p. 66]{Eddington1939}
\end{quote}

- law of Nature (capital) vs law of nature, former about nature herself and latter about observational regularity.  physics enlarged in scope because of subjective and objective parts, what to call it?  science?  Idealism.

\begin{quote}
    It seems to me that the ``enlarged'' physics which is to include the objective as well as the subjective is just \emph{science}; and the objective, which has no reason to conform to the pattern of systematisation that distinguishes present-day physics, is to be found in the non-physical part of science.  We should look for it in the part of biology (if any) which is not covered by biophysics; in the part of psychology which is not covered by psychophysics; and perhaps in the part of theology which is not covered by theophysics.  The purely objective sources of the objective element in our observational knowledge have already been named; they are \emph{life, consciousness, spirit}.
    
    We reach then the position of idealist, as opposed to materialist, philosophy.  The purely objective world is the spiritual world; and the material world is subjective in the sense of selective subjectivism.
    
    \citep[p. 68-69]{Eddington1939}
\end{quote}

\subsection{Chapter V: Epistemology and Relativity Theory}
- ch. 5 Epistemology and Relativity Theory.  quantity and measurement definitions (instrumental)

\begin{quote}
    It has come to be the accepted practice in introducing new physical quantities that they shall be regarded as \emph{defined} by the series of measuring operations and calculations of which they are the result.  Those who associate with the result a mental picture of some entity disporting itself in a metaphysical realm of existence do so at their own risk; physics can accept no responsibility for this embellishment.
    
    \citep[p. 71]{Eddington1939}
\end{quote}

- Poincare on geometry (S\&H), theorist and experimenter must agree to  metrologist definitions, instructions for a procedure of measurement.  fear of experiment, 

\begin{quote}
    The definition of length or distance and the corresponding definition of time-extension are particularly important, because in general the definitions of other physical quantities presuppose that length and time-extension have been defined, and any ambiguity of their meaning would spread through the whole superstructure.  If, instead of length being defined observationally, its definition were left to the pure mathematician, all the other physical quantities would be infected with the virus of pure mathematics.  
    
    Practical physicists have long been occupied with the accurate determination of lengths, and the principles which they strive to follow were settled before the theory of relativity arose.  This branch of practical physics is called metrology.  When therefore it became necessary to adopt formally an observational definition of length, there could be no question of setting up a rival procedure.  The definition must give instructions as to a procedure of measurement of lengths.  To the metrologist these instructions amounted simply to ``Carry on''.
    
    \citep[p. 73]{Eddington1939}
\end{quote}

\begin{quote}
    Accordingly, by length in relativity theory we mean what the metrologist means, not what the pure geometer means.  In accepting relativity principles, the physicist puts aside his paramour pure mathematics, dismisses their go-between metaphysics, and enters into honourable marriage with metrology.  I am afraid those who represent the bride are inclined to suspect that he is not entirely off with his first love.  Some writings on relativity look a bit mathematical.  Since I am not entirely convinced of the innocence of some of my colleagues, I must on this point answer only for myself.  I declare that the suspicions are groundless.  If I sometimes employ pure mathematics, it is only as a drudge; my devotion is fixed on the physical thought which lies behind the mathematics.  Mathematics is a useful vehicle for expression and manipulation; but the heart of the theory is elsewhere:
    
    \begin{quote}
        \begin{center}
            Euphelia serves to grace my measure
            
            But Chloe is my real flame.
        \end{center}
    \end{quote}
    
    The crucial part of the definition of length is the specification of a standard which shall be available for comparison at any place and at any time.  
    
    \citep[p. 74]{Eddington1939}
\end{quote}

- reproducible physical structure standard of length like calcite crystal, quantum structure of matter providing standard of lenght for relativity theory?

\begin{quote}
    Molar physics always has the last word in observation, for the observer himself is molar.
    
    The secret of the union of molar and microscopic physics---of relativity theory and quantum theory---is ``the full circle''.  They are not so much branches forking from one root as semi-circles joined at both ends.  Generally we enter on the circle at the junction now under discussion, where relativity theory takes its standard of length from quantum theory.  But relativity theory, which has made greater progress along its arc than quantum theory along its arc, is already exploring the other junction, where the cosmical constant and matters of that kind are involved.  
    
    \citep[p. 77]{Eddington1939}
\end{quote}

- at least relativity reliance on quantum means they share length standard; constants changing?

\begin{quote}
    It is often suggested that some of the constants of nature, e.g. the velocity of light or the gravitational constant, vary with time.  Unless the standards of length and time-extension have been carefully defined, such discussions are meaningless; and much that has been written on the subject is discounted by the fact that the writers are evidently unaware of the nature of the definition of these standards.  Anyone who suggests variation of a fundamental constant has before him a heavy task of reconstruction of theory and reinterpretation of observational measurements before he can reach any observational confirmation or contradiction of his suggestion.  Meanwhile I think that progress of the epistemological method has assured us that the constants of nature (apart from our arbitrary units) are numbers introduced by our subjective outlook, whose values can be calculated \emph{a priori} and stand for all time.  For this reason my personal conclusion is that there is no more danger that the velocity of light or the constant of gravitation will change with time than that the circumference-diameter ratio $\pi$ will change with time.
    
    Let us examine more closely what is implied in the suggestion that the velocity of light \emph{in vacuo} changes with the time.  An immediate consequence is that the ratio of the wavelength $\lambda$ to the period $T$ of any spectral line, say a hydrogen line, changes with time.  Now for all epochs the standard of time is a time-period in some quantum-specified structure, and the standard of length is a space-extension in some quantum-specified structure.  We may take this structure to be a hydrogen atom in the quantum-specified state in which it emits the line considered.  It follows that either the ratio of the period of the emitted light to the time-period intrinsic in the emitting atom varies with time, or the ratio of the length of the emitted waves to the spatial scale of structure of the emitting atom varies with time.  I do not think those who propose the variability of the velocity of light realise that, if their words have any meaning, they imply that the period of the light has no constant relation to---is therefore not determined by---any corresponding periodicity in its source; or alternatively, that the wavelength of the light has no constant connection with the linear scale of its source.  If this were true, it would involve a conception of atomic structure so far removed from that of present-day quantum theory that scarcely anything in our present knowledge would survive.  
    
    \citep[p. 77-79]{Eddington1939}
\end{quote}

- extreme accuracy, quantum standard of length in strong $E$ or $B$, conventional math theories that approach standard in limit as $E, B$ go to zero

- Section V, does he talk next about something like renormalization?  Talking about a short metric for length (as opposed to long where strain/distortions are greater, shorter is proportionately less strain?)

\begin{quote}
    We cannot always remove the bodies that are causing the strain.  If we are measuring up the solar system, we cannot begin the proceedings by clearing away the sun.  Thus, in general, we have to be content with short standards which are proportionately less affected by strain.  With the short standard we can only measure short distances directly.  To a first approximation we can determine large distances by measuring them in short sections, and summing or integrating the results;\footnote{That is to say, we \emph{define} a large distance as the result of integrating short distances (provided that the result is unambiguous) instead of defining it as the result of a comparison with a long standard.} but to a higher approximation this method also leads to ambiguous results.  This ambiguity is known as the \emph{non-integrability of displacement}.  \citep[p. 82]{Eddington1939}
\end{quote}


- non-integrability of displacement as fundamental to GR

\begin{quote}
    The failure to define long distances observationally, or in mathematical language the non-integrability of displacement, is the foundation of Einstein's theory of gravitation.  According to the usual outlook gravitation is the cause of the trouble; gravitation produces the strains which render long standards useless.  But Einstein's outlook is more nearly that the ``trouble''---the non-integrability of displacement---is the cause of gravitation.  I mean that in Einstein's theory the ordinary manifestations of gravitation are deduced as mathematical consequences of the non-integrability of displacement.  I cannot enter here into the details, which require a large treatise; but the gist of it is that Einstein showed how to specify the non-integrability quantitatively, and used the numbers thus introduced---the famous $g_{\mu\nu}$---as a measure of the influence which disturbs the ideal conditions in which displacements would be integrable.  ``Gravitational field'' is the name which we have given to this influence.  As might be expected, this systematic specification of the gravitational field has been found to be more precise than the casual specification of it by one of its effects which happened to strike Newton's attention when he sat under an apple-tree.  

    Einstein's specification is more accurate than Newton's; but that the two refer to the same thing is seen when we recall that it was the strain, produced by the two ends of the long standard trying to fall with different accelerations towards the sun or moon, which vitiated it as a standard and frustrated our effort to measure directly an integrated length.  We need therefore not be surprised that from Einstein's specification the more ordinary manifestations of gravitation in falling bodies can be deduced.  

    This is a particularly good example of the way in which epistemological study has brought about a great advance in science; and it is worth while to recall the principal steps.  If physics is to describe what we really observe, we must overhaul the definitions of the terms employed in it so that they explicitly refer to observational facts and not to metaphysical conjectures.  Length and time interval in particular need to be carefully defined, since they are the basis of nearly all other physical definitions.  To avoid circular definitions it is essential that the standards of length and time interval should be the extensions of structures completely specified by pure numbers.  With such structures as standard we obtain a definition of infinitesimal intervals (in the absence of an electromagnetic field), but we do not obtain an exact definition of long intervals.  Thus, in order that physics may express purely observational knowledge, it is necessary to develop a system of description of the location of events based wholly on infinitesimal distances and time intervals; we thereby avoid reference to long intervals which have no exact observational definition.  This system of location, depending on infinitesimal intervals, is the foundation of general relativity theory.  In relativity theory a long distance is in general an approximate conception only; it is incapable of exact definition.\footnote{The breakdown of the ordinary definition leaves the term at the disposal of investigators, and various technical definitions of long distances have been proposed.  But these technical uses of the term are irrelevant here.} 
    
    As soon as we realise that the definition of length does not cover long distances and so does not imply integrability of displacement, integrability becomes a special hypothesis which requires defending. One does not accept hypotheses gratuitously.  Proceeding from this rational basis of spacetime measurement we find that the phenomenon of gravitation appears automatically---unless we deliberately introduce a hypothesis of integrability to exclude it---and in this way we are led immediately to Einstein's theory of gravitation.\citep[p. 83-85]{Eddington1939}
\end{quote}

- Quoting the entire section VI of chapter V (Epistemology and Relativity Theory) as it expresses I think fairly clearly an intersubjective idealism about observational physics.  Also note the comment about a subjective interpretation of probability.

\begin{quote}
    I have been continually emphasizing the subjectivity of the universe described in physical science.  But, you may ask, was it not the boast of the theory of relativity that it penetrated beyond the relative (subjective) aspect of phenomena and dealt with the absolute?  For example, it showed that the usual separation of space and time is subjective, being dependent on the observer's motion, and it substituted a four-dimensional space-time independent of the observer.  It may seem difficult to reconcile this view of Einstein's theory as lifting the veil of relativity which hides the absolute from us, with my present account of modern physics as acquiescing in, and making the bet of, a partially subjective universe.

    It is necessary to remember that there has been thirty years' progress.  Relativity began like a new broom, sweeping away all the subjectivity it found.  But, as we have advanced, other influences of subjectivity have been detected which are not so easily eliminated.  Probability, in particular, is frankly subjective, being relative to the knowledge which we happen to possess.  Instead of being swept away, it has been exalted by wave mechanics into the main theme of physical law.

    The subjectivity referred to in these lectures is that which arises from the sensory and intellectual equipment of the observer.  Without varying this equipment, he can vary in position, velocity and acceleration.  Such variations will produce subjective changes in the appearance of the universe to him; in particular the changes depending on his velocity and acceleration are more subtle than was realised in classical theory.  Relativity theory allows us to remove (if we wish) the subjective effects of these \emph{personal} characteristics of the observer; but it does not remove the subjective effects of \emph{generic} characteristics common to all ``good'' observers---although it has helped to bring them to light.

    Confining attention to the personal, as distinguished from the generic, subjectivity, let us see precisely what is meant by removing this subjectivity.  There does not seem to be much difficulty in conceiving the universe as a three-dimensional structure viewed from no particular position; and I suppose we can, after a fashion, conceive it without any standard of rest or of non-acceleration.  It is perhaps rather unfortunate that it is, or seems to be, so easy to conceive; because the conception is liable to be mischievous from the observational point of view.  Since physical knowledge must in all cases be an assertion of the results of observation (actual or hypothetical), we cannot avoid setting up a dummy observer; and the observations which he is supposed to make are subjectively affected by his position, velocity and acceleration.  The nearest we can get to a non-subjective, but nevertheless observational, view is to have before us the reports of all possible dummy observers, and pass in our minds so rapidly from one to another that we identify ourselves, as it were, with all the dummy observers at once.  To achieve this we seem to need a revolving brain.  

    Nature not having endowed us with revolving brains, we appeal to the mathematician to help us.  He has invented a transformation process which enables us to pass very quickly from one dummy observer's account to another's.  The knowledge is expressed in terms of tensors which have a fixed system of interlocking assigned to them; so that when one tensor is altered all the other tensors are altered, each in a determinate way.  By assigning each physical quantity to an appropriate class of tensor, we can arrange that, when one quantity is changed to correspond to the change from dummy observer $A$ to dummy observer $B$, all the other quantities change automatically and correctly.  We have only to let one item of knowledge run through its changes---to turn one handle---to get in succession the complete observational knowledge of all the dummy observers.

    The mathematician goes one step farther; he eliminates the turning of the handle.  He conceives a tensor symbol as containing in itself all its possible changes; so that when he looks at a tensor equation, he sees all its terms changing in synchronised rotation.  This is nothing out of the way for a mathematician; his symbols commonly stand for unknown quantities, and functions of unknown quantities; they are everything at once until he chooses to specify the unknown quantity.  And so he writes down the expressions which are symbolically the knowledge of all the dummy observers at once---until he chooses to specify a particular dummy observer.

    But, after all, this device is only a translation into symbolism of what we have called a revolving brain.  A tensor may be said to symbolise absolute knowledge; but that is because it stands for the subjective knowledge of all possible subjects at once.  

    This applies to personal subjectivity.  To remove the generic subjectivity, due say to our intellectual equipment, we should have similarly to symbolise the knowledge as it would be apprehended by all possible types of intellect at once.  But this could scarcely be accomplished by a mathematical transformation theory.  And what would be the result if it were accomplished?  According to Chapter IV, if we remove all subjectivity we remove all the fundamental laws of nature and all the constants of nature.  But, after all, these subjective laws and facts happen to be important to beings who are not equipped with revolving brains and variable intellects.  And if the physicist does not take charge of them, no one else is qualified to do so.  Even in relativity theory, which deals with the absolute (in a somewhat limited sense), we continually hark back to the relative to examine how our results will appear in the experience of an individual observer.  We are not so eager now as we were twenty years ago to eliminate the observer from our world view.  Sometimes it may be desirable to banish him and his subjective distortion of things for a time, but we are bound to bring him back in the end; for he stands for---ourselves. \citep[p. 85-88]{Eddington1939}
\end{quote}

\subsection{Chapter VI: Epistemology and Quantum Theory}
- ch 6, Epistemology and Quantum Theory

\begin{quote}
    I must still keep hammering at the question, What do we really observe?  Relativity theory has returned one answer---we only observe \emph{relations}.  Quantum theory returns another answer---we only observe \emph{probabilities}. \citep[p. 89]{Eddington1939}
\end{quote}

- Heisenberg uncertainty principle as consequence of different measurements as source for irreducible indeterminism in QM

\begin{quote}
    The suggestion is that in the new physics the so-called probabilities are actually the real entities---the elemental stuff of the physical universe.  We have precise knowledge of \emph{them}; and it would seem retrogressive to postulate other entities behind them of which our knowledge must always be uncertain.  

    I think that this idea is at the back of a rather common suggestion that a proper reformulation of our elementary concepts would banish the present indeterminism from the system of physics.  The idea is that the indeterminism revealed by the new physics is not intrinsic in the universe, but appears only in our attempt to connect it with the obsolete universe of classical physics.  Probability would then be merely the funnel through which the new wine is poured into old bottles.  

    But the suggestion overlooks the essential feature of the indeterminism of the present system of physics, namely that the quantities which it can predict only with uncertainty are quantities which, \emph{when the time comes}, we shall be able to observe with high precision.  The fault is therefore not in our having chosen concepts inappropriate to observational knowledge.  For example, Heisenberg's principle tells us that the position and velocity of an electron at any moment can only be known with a mutually related uncertainty; and, taking the most favourable combination, the position of the electron one second later is uncertain to about 4 centimetres.  This is the uncertainty of the prediction from the best possible knowledge we can have at the time.  But one second later the position can be observed with an uncertainty of no more than a fraction of a millimetre.  It has often been argued that the impossibility of knowing simultaneously the exact position and exact velocity only shows that position and velocity are unsuitable conceptions to use in expressing our knowledge.  I have no special attachment to these conceptions; and I will grant, if you like, that our knowledge of the universe at the present moment can be regarded as perfectly determinate (the supposed indeterminacy being introduced in translating it into an inappropriate frame of conception).  But that does not remove the ``indeterminism'' (which is distinct from the ``indeterminacy''), namely that this knowledge, however expressed, is inadequate to predict quantities which, independently of our frame of conception, can be directly observed when the time comes.  
    
    Returning to the more general aspect of the probability conception, we find that it cannot be got rid of by any transformation of outlook.  It is not possible to transform the current system of physics, which by its equations links probabilities in the future with probabilities in the present, into a system which links ordinary physical quantities in the future with ordinary physical quantities in the present, without altering its observable content.  The bar to such a transformation is that probability is not an ``ordinary physical quantity''.  At first sight it appears to be one; we obtain knowledge of it from observation, or from a mixture of observation and deduction, as we obtain knowledge of other physical quantities.  But it is differentiated from them by a peculiar irreversibility of its relation to observation.  The result of an observation determines definitely a probability distribution of some quantity, or a modification of a previously existing probability distribution; but the connection is not reversible, and a probability distribution does not determine definitely the result of an observation.  For an ordinary physical quantity there is no difference between making a new determination and verifying a predicted value; but for probability the procedures are distinct.  

    Thus we may expand the answer of quantum theory that ``we only observe probabilities'' into the form: The synthesis of knowledge which constitutes theoretical physics is connected with observation by an \emph{irreversible} relation of the formal type familiar to us in the concept of probability.
    \citep[p. 90-91]{Eddington1939}
\end{quote}

- End of quote above talking about probability as structural invariant in some sense like Cassirer would general covariance (does Ryckman also notice this?)

- Irreversibility illustrated through bag w/ colored ball example, comparing to wave packet probability distributions

\begin{quote}
    According to wave mechanics, an observation determines or produces a concentrated wave packet in the probability distribution.  This wave packet diffuses according to laws embodied in the equations of the theory; and we can calculate the form into which the wave packet will have spread one unit of time later.  \emph{But the theory does not assert that this is the form of wave packet which would be produced by an observation made one unit of time later.}  On the other hand, if from the observationally determined form of a sound wave at one instant we calculate the form into which it will have spread one unit of time later, the whole point of the theory is that we obtain the form which would be determined by observations made one unit of time later.  \citep[p. 93]{Eddington1939}
\end{quote}

- ``an observation'' vs ``an item of observational knowledge''.  Holistic view, Quine?

\begin{quote}
    More generally we must recognise that an item of observational knowledge involves, besides a primary pointer reading, secondary pointer readings identifying the circumstances in which the primary pointer reading occurred.  It must be admitted that even an isolated pointer reading is an item of knowledge of a sort; but it is not with such items that the scientific method deals.  For scientific knowledge the association with other pointer readings is an essential condition; and we may therefore describe physical knowledge as a knowledge of the associations of pointer readings.  \citep[p. 100]{Eddington1939}
\end{quote}

- Referring to other book, criticizing older position.  Alluding to uncertainty principle: pointer reading interference.  Wave mechanics(!)  Scope of physical knowledge and defining physical universe.

\begin{quote}
    In \emph{The Nature of the Physical World} it is emphasised that physical knowledge is concerned with the connection of pointer readings rather than with the pointer readings themselves; and it is concluded that the connectivity of pointer readings, as expressed by the laws of physics, supplies the common background which realistic problems always demand---the background described by the tertiary pointer readings which are not determined afresh for each individual item of knowledge.  But, if I may venture to criticise the author of that book, he does not seem to have appreciated the difficulty which arises through the interference of pointer readings with one another when we contemplate such an unlimited multiplicity of pointer readings.  It is true that the interference is negligible in molar physics (to which the discussion in \emph{The Nature of the Physical World} was limited).  But in a fundamental discussion of this kind it is not legitimate to separate molar physics from microscopic physics; for we have seen (p. 76) that neither branch is logically complete in itself.  
    
    Our definition of the physical universe is that it is the world which physical knowledge is formulated to describe.  The interference of observations creates a difficulty which must be met in one of two ways.  Either we must take the complete description of the physical universe to embody more than the totality of our possible knowledge of it; so that, whichever of two interfering observations we choose to make, there will be a place for it in the description.  Or we must adopt a \emph{flexible} universe containing nothing which is not represented by our actual knowledge (or in theoretical discussions by the supposedly actual knowledge furnished as data of the problem considered).  In the first alternative we cannot consistently suppose all the items of the complete description to be represented by actual pointer readings; and it is therefore not true to say that its structure is a connectivity of pointer readings.  The second alternative is adopted in wave mechanics, which accepts as leading features of the physical universe the probability waves created by actual observation of the physical quantities with which they are associated.  Clearly there is no more than a formal distinction between the study of a universe flexible according to the knowledge we happen to have of it and a direct study of the knowledge itself.  Either alternative brings us back to the conclusion that the common background is required to connect one item of knowledge with the rest of knowledge, rather than one element of an external universe with the rest of the universe.
    \citep[p. 100-102]{Eddington1939}
\end{quote}

- Eddington provides a summary (section V of ch. VI Epistemology and Quantum Theory) which will be useful to have (I feel like I'm typing the whole book...)  Adapting very slightly the enumeration to be LaTeX friendly

\begin{quote}
    The following summary will recall the principal conclusions that we have so far reached:
    \begin{enumerate}
        \item Physical knowledge (by definition) includes only knowledge capable of observational test; an item of physical knowledge must therefore assert the result of a specified observational procedure.
        \item The definitions of the terms used in expressing physical knowledge must be such as to secure that (1) is satisfied.  In particular the definition of a physical quantity must specify unambiguously a method of measuring it.  
        \item Strict adherence to (2) involves a number of modifications of the conceptions and practice of classical physics; and indeed there still survive glaring violations of it in current quantum theory.  The points (4) to (9) below arise when the definitions are scrutinised from this point of view.
        \item The first definitions required are those of length and time-interval, since the definitions of other physical quantities presuppose these.  The standards of length and time must be structures specified by pure numbers only (since no other quantitative terms are available at this early stage).  This means that the standards must be reproducible from a quantum specification.
        \item Only short standards, suitable for measuring infinitesimal displacements in space and time, are provided by such specifications; and it must not be assumed that the infinitesimal displacements so measured are integrable.
        \item Owing to the interference of exact observations with one another, an attempt to define observationally the exact conditions under which the measurement of a physical quantity is intended to be carried out breaks down.  It is therefore necessary to leave the minor details to chance.
        \item In this way the probability conception is incorporated in the fundamental definitions.  It introduces an irreversible relation between observation and formulated observational knowledge.  This irreversibility makes the existing system of physics indeterministic, considered as a system of prediction of what can be observed at a future time.
        \item Certain quantities used in the formulation of physical  knowledge in classical physics are found to have no definition satisfying (2).  These are unobservables, e.g. absolute simultaneity at a distance.
        \item Other quantities, conditionally observable, have been employed in conditions in which they are unobservable.  For example, the definition of relative coordinates presupposes that the particles are distinguishable, but ordinary relative coordinates are still used erroneously in problems concerning indistinguishable particles.
        \item The conclusions (4) to (9) are reached by considering the way in which physical knowledge is obtained and formulated. We refer to them as epistemological or \emph{a priori} conclusions, to distinguish them from \emph{a posteriori} conclusions derived from a study of the results of observations which have been obtained and formulated in this way.
        \item Although epistemological conclusions are of the nature of truisms, they have far-reaching consequences in physics.  Thus the unobservability of absolute simultaneity (8) leads to the special theory of relativity; the non-integrability of displacement (5) leads to Einstein's theory of gravitation; the introduction of the probability conception in a fundamental way (7) leads to the method of wave mechanics.
        \item In the modified theories which result, epistemological principles play a part which was formerly taken by physical hypotheses, i.e. generalisations suggested by an \emph{a posteriori} study of the results of observation.
        \item Current relativity theory and quantum theory, as usually accepted, have not yet taken full advantage of this epistemological method.  It appears that when the epistemological scrutiny of definitions is systematically applied, and its consequences are followed up mathematically, we are able to determine all the ``fundamental'' laws of nature (including purely numerical constants of nature) without any physical hypothesis.
        \item This means that the fundamental laws and constants of physics are wholly subjective, being the mark of the observer's sensory and intellectual equipment on the knowledge obtained through such equipment; for we could not have this kind of \emph{a priori} knowledge of laws governing an objective universe.
        \item It is not suggested that the physical universe is wholly subjective.  Physical knowledge comprises, besides ``laws of nature'', a vast amount of special information about the particular objects surrounding us.  This information is doubtless partly objective as well as partly subjective.
        \item The subjective laws are a consequence of the conceptual frame of thought into which our observational knowledge is forced by our method of formulating it, and can be discovered \emph{a priori} by scrutinising the frame of thought as well as \emph{a posteriori} by examining the actual knowledge which has been forced into it.
        \item The characteristic form of the fundamental laws of physics is the stamp of subjectivity.  If there are also laws of objective origin, they may be expected to be of a different type.  It seems probable that wherever effects of objective governance have appeared they have been regarded as an indication that the subject is ``outside physics'', e.g. conscious volition, or possibly life.
        \item Epistemological laws (if correctly deduced) are compulsory, universal, and exact.  Since the fundamental laws of physics are epistemological, they have this character---contrary to the view usually advocated in scientific philosophy, which has assumed that they are merely empirical regularities.\citep[p. 102-105]{Eddington1939}
    \end{enumerate}
\end{quote}

\subsection{Chapter VII: Discovery or Manufacture?}

- Ch. VII Discovery or Manufacture?  About limits of measurement.  Fourier/spectroscopic analysis of (white) light.  Also discussed Rutherford experiment (!), which of course is one of the topics of Larson's Case Against the Nuclear Atom.

\begin{quote}
    The realisation that natural white light is a quite irregular disturbance, into which regularity is introduced by our method of spectroscopic examination of it, was the first sign of an uneasiness among physicists as to whether in our experiments we may not interfere so much as to destroy what we were seeking to investigate.  The uneasiness has become more acute in modern atomic physics, since we have no tool fine enough to probe an atom without grossly disturbing it.  

    The question I am going to raise is---how much do we discover and how much do we manufacture by our experiments?  When the late Lord Rutherford showed us the atomic nucleus, did he \emph{find} it or did he \emph{make} it?  It will not affect our admiration of his achievement either way---only we should rather like to know which he did.  The question is one that scarcely admits of a definite answer.  It turns on a matter of expression, like the question whether the spectroscope finds or whether it makes the green colour which it shows us.  But since most people are probably under the impression that Rutherford found the atomic nucleus, I will make myself advocate for the view that he made it.

    \citep[p. 108-109]{Eddington1939}
\end{quote}

- WAVES WAVES WAVES.  Form.  Advocating for the "make" view.  Continuing on observer effect (physical interaction) but also how subjective selection makes.  Procrustes.  Observer effect stuff.  Questioning neutrinos.

\begin{quote}
    The tendency of writers on quantum theory has been perhaps to go farther than I do in emphasising the \emph{physical} interference of our experiments with the objects which we study.  It is said that the experiment puts the atoms or the radiation into the state whose characteristics we measure.  I shall call this Procrustean treatment.  Procrustes, you will remember, stretched or chopped down his guests to fit the bed he had constructed.  But perhaps you have not heard the rest of the story.  He measured them up before they left next morning, and wrote a learned paper ``On the Uniformity of Stature of Travellers'' for the Anthropological Society of Attica.

    The physical violence, however, is not really the essential point.  Ideally the experimenter might wait until the conditions of his experiment happened naturally, as those engaged in the observational sciences are forced to do.  We grossly interfere with the irregularity of white sunlight by passing it through a spectroscope; but sunlight may occasionally fall through a crevice on to a natural crystal and form a spectrum without our help.  The standard conditions, which turn aimless measurement into a good measurement of a definite physical quantity useful for scientific induction, may sometimes occur without human interference.  But, so far as physical theory is concerned, it makes no difference whether we \emph{create} or whether we \emph{select} the conditions which we study.  Whether the interference of the observer is physical or selective, it is none the less marked in the resulting conclusions.  The kind of observation on which physical theory is based is not a casual taking notice of things around us, nor a general running round with a measuring rod.  Under cover of the term ``good'' observation the bed of Procrustes is artfully concealed.

    To what length can this interference be carried?  I do not think that any limit can be set \emph{a priori}.  It is pertinent to remember that the concept of substance has disappeared from fundamental physics; what we ultimately come down to is \emph{form}.  Waves!  Waves!!  Waves!!!  Or for a change---if we turn to relativity theory---curvature!  Energy which, since it is conserved, might be looked upon as the modern successor of substance, is in relativity theory a curvature of space-time, and in quantum theory a periodicity of waves.  I do not suggest that either the curvature or the waves are to be taken in a literal objective sense; but the two great theories, in their efforts to reduce what is known about energy to a comprehensible picture, both find what they require in a conception of ``form''.  

    Substance (if it had been possible to retain it as a physical conception) might have offered some resistance to the observer's interference; but form plays into his hands.  Suppose an artist puts forward the fantastic theory that the form of a human head exists in a rough-shaped block of marble.  All our rational instince is roused against such an anthropomorphic speculation.  It is inconceivable that Nature should have placed such a form inside the block.  But the artist proceeds to verify his theory experimentally---with quite rudimentary apparatus too.  Merely using a chisel to separate the form for our inspection, he triumphantly proves his theory.  Was it in this way that Rutherford rendered concrete the nucleus which his scientific imagination had created?

    Do not be misled by thinking of the nucleus as a sort of billiard ball.  Think of it rather as a system of waves.  It is true that the term ``nucleus'' is not strictly applicable to the waves (cf. the electron, p. 51): but it is equally unrigorous to speak of the nucleus as having been ``discovered''.  The discovery does not go beyond the waves which represent the knowledge we have of the nucleus.

    Does the sculptor's procedure differ in any essential way from that of the physicist?  The latter has a conception of a harmonic wave form which he sees in the most unlikely places---in irregular white light, for example.  With a grating instead of a chisel, he separates it from the rest of the white light and presents it for our inspection.  Just as the sculptor separates the rough block of marble into a bust and a heap of chips, so the physicist separates the irregular wave disturbance into a simple harmonic green wave and a scrap-heap of other components.  In Fourier and other recognised methods of analysis, physics allows and practises the splitting of form into components.  It allows us to select a form which \emph{we ourselves} have prescribed, and treat the rest as contamination which we can remove, if we can devise the necessary apparatus, so as to exhibit our selected form by itself.  In every physical laboratory we see ingeniously devised tools for executing the work of sculpture, according to the designs of the theoretical physicist.  Sometimes the tool slips and carves off an odd-shaped form which we had not expected.  Then we have a new experimental discovery.

    It is difficult to see where, if at all, a line can be drawn.  The question does not merely concern light waves, since in modern physics form, particularly wave form, is at the root of everything.  If no line can be drawn, we have the alarming thought that the physical analyst is an artist in disguise, weaving his imagination into everything---and unfortunately not wholly devoid of the technical skill to realise his imagination in concrete form.

    An illustration may show that a serious practical question is raised.  Just now nuclear physicists are writing a great deal about hypothetical particles called \emph{neutrinos} supposed to account for certain peculiar facts observed in $\beta$-ray disintegration.  We can perhaps best describe the neutrinos as little bits of spin-energy that have got detached.  I am not much impressed by the neutrino theory.  In an ordinary way I might say that I do not believe in neutrinos.\footnote{Doubtless until a truer understanding of the spin problem is reached, it is better to make shift with neutrinos than to ignore the difficulty which they are intended to meet.  I have no objection to neutrinos as a temporary expedient, but I would not expect them to survive---except that, as suggested in this paragraph, survival may not be wholly a question of intrinsic merit.}  But I have to reflect that a physicist may be an artist, and you never know where you are with artists.  My old-fashioned kind of disbelief in neutrinos is scarcely enough.  Dare I say that experimental physicists will not have sufficient ingenuity to \emph{make} neutrinos?  Whatever I may think, I am not going to be lured into a wager against the skill of experimenters under the impression that it is a wager against the truth of a theory.  If they succeed in making neutrinos, perhaps even in developing industrial applications of them, I suppose I shall have to believe---though I may feel that they have not been playing quite fair.  

    The question is raised whether the experimenter really provides such an effective control on the imagination of the theorist as is usually supposed.  Certainly he is an incorruptible watch-dog who will not allow anything to pass which is not observationally true.  But there are two ways of doing that---as Procrustes realised.  One is to expose the falsity of an assertion.  The other is to alter things a bit so as to make the assertion true.  And it is admitted that our experiments \emph{do} alter things.  

    I have been acting as advocate for an extreme view, presuming that your natural prejudices are all the other way.  I must now try to recover the poise of a judge.  I do not think that \emph{as yet} the analytical imagination of the mathematical physicist has developed into the unfettered imagination of the artist.  He plays the game according to certain rules which, arbitrary as they may seem at first sight, express an epistemological principle that goes deep into the roots of human thought.  This we shall discuss presently.  But have we a guarantee that the rules are for all time?  The boy who outrageously breaks the rules of a game may be suitably punished by his companions, or he may be commemorated as the founder of Rugby football.  The man who makes neutrinos will not be punished if he has overstepped the rules; he will be acclaimed for freeing physics from an obstruction to its useful development.

    \citep[p. 109-113]{Eddington1939}
\end{quote}


\subsection{Chapter VIII: The Concept of Analysis}


- ch. VIII Concept of Analysis.  Intellectual activity.  Objects of knowledge filtered through frame of thought.  Parts require set of parts.

\begin{quote}
    To explain why we have to start from the notion of a complete set of parts rather than from the apparently simpler notion of a single part, I must ask a question.  Is the bung-hole of a barrel part of the barrel?  Think well before you answer; because the whole structure of theoretical physics is trembling in the balance.  \citep[p. 119]{Eddington1939}
\end{quote}

\begin{quote}
    Next suppose that we adhere to Euclid's axiom and decide that the bung-hole of a barrel is not part of the barrel.  The objection to this is that it has long ceased to be the form of thought employed in physics.  It is, I think, really a compound association of two concepts, the concept of analysis and the concept of substance.  The concept of substance introduces a clear distinction of positive and negative; so that we can have a limited form ofthe concept of analysis, which we may call substance-analysis, in which the systems of analysis are restricted to those which furnish a complete set of \emph{positive} parts.  When the analysis is not associated with substance (or with a structurally equivalent concept), when for example it is associated with wave form, the restriction cannot be imposed.  In optics darkness is considered to be constituted of two interfering light waves; light may be a ``part'' of darkness.  In Fourier analysis the components partially cancel one another in the manner of positive and negative quantities.  Thus, although there may be cases in physics in which analysis is applied to entities which by definition are essentially positive and the restriction of substance-analysis applies, we now look on it as an incidental restriction in a particular application and not as part of the fundamental concept of analysis.  

    That the general form of the concept of analysis is the form accepted in physical science is shown conclusively by the example of the positron.  A positron is a hole from which an electron has been removed; it is a bung-hole which would be evened up with its surroundings if an electron were inserted.  But it would be out of the question nowadays to define ``part'' in such a way that electrons are parts of a physical system but positrons are not.  

    You will see that the physicist allows himself even greater liberty than the sculptor (p. 111). The sculptor removes material to obtain the form he desires.  The physicist goes further and adds material if necessary---an operation which he describes as removing negative material.  He fills up a bung-hole, saying that he is removing a positron.  But he still claims that he is only revealing---sorting out---something that was already there.  \citep[p. 120-121]{Eddington1939}
\end{quote}


\begin{quote}
    Our purpose is to expose, not necessarily to justify, the frame of thought underlying the expression of our physical knowledge.  Partially at least we emancipate ourselves from a frame of thought as soon as we realise that it is only a frame of thought and not an objective truth we are accepting.  Any power for mischief it may have is sterilised so long as it is kept exposed.  I would not like to say that the concept of analysis is a necessity of thought.  But, whether it is a necessary form or not, it has dominated the development of present-day physics, and we have to follow up its influence on the scheme of description of phenomena which has resulted.

    \citep[p. 121]{Eddington1939}
\end{quote}

- analysis (reduction?) breeding sameness, inevitability of structurally same units, concept of same unit of structure, impress of form of thought, electron vs. protons, quantum

\begin{quote}
    I will call this specialisation of the concept of analysis the \emph{atomic concept}, or for greater precision the \emph{concept of identical structural units}.

    The new conception is, not merely that the whole is analysable into a complete set of parts, but that it is analysable into parts which resemble one another.  It is at the opposite pole from the analysis, say, of a human being into soul and body, in which the two parts belong to altogether different categories of entities.  I will go farther, and say that the aim of the analysis employed in physics is to resolve the universe into structural units which are \emph{precisely} like one another.

    It may be objected that the structural units recognised in present-day physics, though resembling one another to a certain extent, are not precisely alike.  The Fourier components of white light, though all simple harmonic trains of waves, differ in wave-length---a difference which we observe as difference of color.  But this difference is not intrinsic.  It depends on the relation of the observer to the structural unit; if he recedes from the source of light, green light turns to red.  Intrinsically the constituents of light---the wave trains or the photons---are all precisely alike; it is only in their relations to the observer, or to external objects generally, that they differ.  That is the essence of relativity theory.  All the variety in the world, all that is observable, comes from the variety of relations between entities.  Therefore when we reach the consideration of the intrinsic nature or structure of the entities that are related, there is nothing left but sameness---in so far as that nature or structure comes within the scope of physical knowledge and is part of the universe which physical knowledge describes.  

    Granting that the elementary units found in our analysis of the universe are precisely alike intrinsically, the question remains whether this is because we have to do with an objective universe built of such units, or whether it is because our form of thought is such as to recognise only systems of analysis which shall yield parts precisely like one another.  Our previous discussion has committed us to the latter as the true explanation.  We have claimed to be able to determine by \emph{a priori} reasoning the properties of the elementary particles recognised in physics---properties confirmed by observation.  This would be impossible if they were objective units.  Accordingly we account for this \emph{a priori} knowledge as purely subjective, revealing only the impress of the equipment through which we obtain knowledge of the universe and deducible from a study of the equipment.  We now say more explicitly that it is the impress of our frame of thought on the knowledge forced into the frame.

    We have just seen that the concept of identical structural units is implicit in the relativity outlook, which attributes variety to relations and not to intrinsic differences in the relata; but I suppose it would be too much to claim that the relativity outlook is engrained in us---that our minds are so constituted that we cannot help moulding our thoughts in the Einsteinian way.  I want to show therefore that the concept of identical structural units expresses a very elementary and instinctive habit of thought, which has unconsciously directed the course of scientific development.  Briefly, it is the habit of thought which regards variety always as a challenge to further analysis; so that the \emph{ultimate} end-product of analysis can only be sameness.  We keep on modifying our system of analysis until it is such as to yield the sameness which we insist on, rejecting earlier attempts (earlier physical theories) as insufficiently profound.  The sameness of the ultimate entities of the physical universe is a foreseeable consequence of forcing our knowledge into this form of thought.  That it is really engrained in us can be seen from the following example.  

    Analysis of matter, as usually presented in present-day theory, reaches a considerable degree of homogeneity of the ultimate parts, but does not quite attain the ideal.  We find protons exactly like one another; we also find electrons, like one another but differing from protons.  Thus the physicist recognises two varieties of elementary units; and nowadays it is difficult to restrain him from adding several others.  Why does a proton differ from an electron?  The answer suggested by relativity theory is that they are actually similar units of structure, and the difference arises in their relations to the general distribution of matter which forms the universe.  The one is related right-handedly and the other left-handedly.  This accounts for the difference of charge; and the difference of mass is also (in a more complicated way) a difference of relation to the external matter without which there would be no means of determining mass observationally.  There is no reasonable doubt that this answer is correct; but what interests us here is not the scientific answer resulting from the application of relativity theory, but the way in which we instinctively try to account for the difference.  We cannot allow ourselves to think of the difference between a proton and an electron as an irreducible dualism---like the difference between soul and body.  (I use the best comparison I can find; but the form of thought, which insists on getting behind---on explaining---variety, is so universal that even the dualism of soul and body is challenged by it.)  No sooner do we discover a difference between protons and electrons than we begin to wonder what makes them different.  When this question arises, we always fall back on structure.  We try to explain the difference as a difference of structure, the structure of the proton being presumably the more complicated.  But if protons and electrons possess structure, they cannot be the ultimate units of which structure is built.  Therefore the present variety of the end-products of physical analysis is an indication that we have not yet touched bottom; and we must push our investigations farther, till we reach identical units which will not challenge us to farther analysis.  The inference, as it happens, is fallacious, because the difference between protons and electrons is in the external relations and is not intrinsic.  But a fallacious inference is informative as to our background of thought; and the thought which insists on intruding is that things which differ do so because they have different structure.  The difference resides in the structure and not in the units out of which structure is built.

    I conclude therefore that our engrained form of thought is such that we shall not rest satisfied until we are able to represent all physical phenomena as an interplay of a vast number of structural units intrinsically alike.  All the diversity of phenomena will then be seen to correspond to different forms of relatedness of these units or, as we should usually say, different configurations.  There is nothing in the external world which dictates this analysis into similar units, just as there is nothing in the irregular vibrations of white light which dictates our analysis of it into monochromatic wave trains.  The dictation comes from our own way of thought which will not accept as final any other form of solution of the problem presented by sensory experience.

    In current quantum theory the analysis approaches, but has not yet reached, this ideal.  For that reason quantum physicists are still unsatisfied that they have got to the bottom of the relationships of the various kinds of particle that they recognise, and of the connection between gravitation, electromagnetism and quantisation.  

    \citep[p. 122-126]{Eddington1939}
\end{quote}

- he then refers to p. 162-169

- section IV, self sufficiency, interaction

\begin{quote}
    The conception of permanently self-sufficient parts of the physical universe is self-contradictory; for such parts are necessarily outside observational knowledge, and therefore not part of the universe which observational knowledge is formulated to describe.  

    The model structure of an atom is incomplete unless it contains some provision by which \emph{we} can become aware of what is happening in the atom.  In short, physics having taken the world to pieces, has the job of cementing it together again.  The cement is called \emph{interaction}.

    \citep[p. 127]{Eddington1939}
\end{quote}



% FILL IN HERE



\subsection{Chapter IX: The Concept of Structure}








\begin{quote}
    Properly to realise the conception of group-structure, we must think of the pattern of interweaving as abstracted altogether from the particular entities and relations that furnish the pattern.  In particular, we can give an exact mathematical description of the pattern, although mathematics may be quite inappropriate to describe what we know of the nature of the entities and operations concerned in it.  In this was mathematics gets a footing in knowledge which intrinsically is not of a kind suggesting mathematical conceptions.  Its function is to elucidate the group-structure of the elements of that knowledge.  It dismisses the individual elements by assigning to them symbols, leaving it to non-mathematical thought to express the knowledge, if any, that we may have of what the symbols stand for.
    
    We shall refer to this abstraction as the mathematical concept of structure, or briefly as the \emph{concept of structure}.  Since the structure, abstracted from whatever possesses the structure, can be exactly specified by mathematical formulae, our knowledge of structure is communicable, whereas much of our knowledge is incommunicable.  I cannot convey to you the vivid knowledge which I have of my own sensations and emotions.  There is no way of comparing my sensation of the taste of mutton with your sensation of the taste of mutton; I can only know what it tastes like to me, and you can only know what it tastes like to you.  But if we are both looking at a landscape, although there is no way of comparing our visual sensations as such, we can compare the \emph{structures} of our respective visual impressions of the landscape.  It is possible for a group of sensations in my mind to have the same structure as a group of sensations in your mind.  It is possible also that a group of entities which are not sensations in anyone's mind, associated together by relations of which we can form no conception, may have this same structure.  We can therefore have structural knowledge of that which is outside everyone's mind.  This knowledge will consist of the same kind of assertions as those which are made about the physical universe in the modern theories of mathematical physics.  For strict expression of physical knowledge a mathematical form is essential, because that is the only way in which we can confine its assertions to structural knowledge.  Every path to knowledge of what lies beneath the structure is then blocked by an impenetrable mathematical symbol.
    
    Physical science consists of purely structural knowledge, so that we know only the structure of the universe which it describes.  This is not a conjecture as to the nature of physical knowledge; it is precisely what physical knowledge as formulated in present-day theory states itself to be.  In fundamental investigations the conception of group-structure appears quite explicitly as the starting point; and nowhere in the subsequent development do we admit material not derived from group-structure.
    
    The fact that structural knowledge can be detached from knowledge of the entities forming the structure, gets over the difficulty of understanding how it is possible to conceive a knowledge of anything which is not part of our own minds.  So long as the knowledge is confined to assertions of structure, it is not tied down to any particular realm of content.  It will be remembered that we have separated the question of the nature of knowledge from the question of assurance of its truth.  We are not here considering how it is possible to be assured of the truth of knowledge relating to something outside our minds; we are occupied with the prior question how it is possible to make any kind of assertion about things outside our minds, which (whether true or false) has a definable meaning.
    
    \citep[p. 141-143]{Eddington1939}
\end{quote}

\begin{quote}
    What sort of thing is it that I know?  The answer is \emph{structure}.  To be quite precise, it is structure of the kind defined and investigated in the mathematical theory of groups.
    
    \citep[p. 147]{Eddington1939}
\end{quote}

\begin{quote}
    The bewilderment of the philosophers evidently arises from a belief that, if we start from zero, any knowledge of the external world must begin with the assumption that a sensation makes us aware of something in the external world---something differing from the sensation itself because it is non-mental.  But knowledge of the physical universe does not begin in that way.  One sensation (divorced from knowledge already obtained by other sensations) tells us nothing; it does not even hint at anything outside the consciousness in which it occurs.  The starting point\footnote{I mean the logical starting point, not the historical starting point, of a subject which has grown out of crude beginnings} of physical science is knowledge of \emph{the group-structure of a set of sensations} in a consciousness.  When these fragments of structure, contributed at various times and by various individuals, have been collated and represented according to the forms of thought that we have discussed, and when the gaps have been filled by an inferred structure depending on the regularities discovered in the directly known portions, we obtain the structure known as the physical universe.
    
    \citep[p. 147-148]{Eddington1939}
\end{quote}


\begin{quote}
    The recognition that physical knowledge is structural knowledge abolishes all dualism of consciousness and matter.  Dualism depends on the belief that we find in the external world something of a nature incommensurable with what we find in consciousness; but all that physical science reveals to us in the external world is group-structure, and group-structure is also to be found in consciousness.  When we take a structure of sensations in a particular consciousness and describe it in physical terms as part of the structure of an external world, it is still a structure of sensations.  
    
    \citep[p. 150]{Eddington1939}
\end{quote}

\begin{quote}
    Let us denote by $X$ the entity of which the physical universe is the structure,\footnote{I usually call $X$ the ``external world'', the ``physical world'' being limited to the structure of the external world.} and distinguish the small part $X_s$ known to be of sensory nature from the remainder $X_u$ of which we have no direct awareness.  It may be suggested that there remains a dualism of $X_s$ and $X_u$ equivalent to the old dualism of consciousness and matter; but this is, I think, a logical confusion, involving a switch over from the epistemological view of the universe as the theme of knowledge to an existential view of the universe as something of which we have to obtain knowledge.  Structurally $X_u$ is no different from $X_s$, and to give meaning to the supposed dualism we have to imagine a supplementary non-structural knowledge of $X_u$ revealing its unlikeness to $X_s$.  We have to suppose that a direct awareness of $X_u$, if we could possess it, would show that it was not of sensory nature.  But the supposition is nonsense; for if we had the supposed direct awareness of $X_u$, it would \emph{ipso facto} be a sensation in our consciousness.  Thus we cannot give meaning to the dualism without making a supposition which eliminates the dualism.

    Although the statement that the universe is of the nature of a ``thought or sensation in a universal Mind'' is open to criticism, it does at least avoid this logical confusion.  It is, I think, true in the sense that it is a logical consequence of the form of thought which formulates our knowledge as a description of a universe.  But it requires more guarded expression if it is to be accepted as a truth transcending forms of thought.
    
    To sum up.  The physical universe is a structure.  Of the $X$ of which it is the structure, we only know that $X$ includes sensations in consciousness.  To the question: What is $X$ when it is not a sensation in any consciousness known to us? the right answer is probably that the question is a meaningless one---that a structure does not necessarily imply an $X$ of which it is the structure.  In other words, the question takes us to a point where the form of thought in which it originates ceases to be useful.  The form of thought can only be preserved by still attributing to $X$ a sensory nature---a sensation in a consciousness unknown to us.  What interests us is not the positive conslusion, but the fact that in no circumstances are we required to contemplate an $X$ of non-sensory nature.  \citep[p. 150-151]{Eddington1939}
\end{quote}

- then Russell


\subsection{Chapter X: The Concept of Existence}

- disagreement over what exists, overdraft example

- GO BACK HERE AND RE-READ AND GET QUOTES

-idempotence

uranoid

\subsection{ChapterXI: The Physical Universe}

\begin{quote}
    I believe there are 15,747,724,136,275,002,577,605,653,961,181,555,468,044,717,914,527,116,709,366,231,425,076,185,631,031,296 protons in the universe, and the same number of electrons.  \citep[p. 170]{Eddington1939}
\end{quote}


- omelette analogy, uncountable particles

\begin{quote}
    A nucleus is composed of protons and electrons in the same way that an omelette is composed of eggs; that is to say, when the omelette appears on the table, there are fewer eggs in the larder.  The proton-electron composition assigned to the various nuclei is amply confirmed by transmutation experiments which apply directly the ``omelette'' criterion of composition.  I think the craze for a metaphysical pronouncement, that eggs and electrons cease to exist when they are scrambled, has now died away; but it was in any case an irrelevancy.  \citep[p. 170-171]{Eddington1939}
\end{quote}

- degrees of belief, uncountable, 

\begin{quote}
    One believes with varying degrees of confidence.  My belief that I know the exact number of protons and electrons in the universe does not rank among my strongest scientific convictions, but I should describe it as a fair average sort of belief.  I am, however, strongly convinced that, if I have got the number wrong, it is just a silly mistake, which would speedily be corrected if there were more workers in this field.  In short, to know the exact number of particles in the universe is a perfectly legitimate aspiration of the physicist.  \citep[p. 171]{Eddington1939}
\end{quote}

\begin{quote}
    Let us see why protons and electrons are uncountable.  It is not merely because there are so many of them.  Quantum physicists tell us that an electron is not definitely in one place but is smeared over a probability distribution; also that electrons are indistinguishable from one another.  That is not very promising material for counting.  There is nothing to remember about the electron you last counted---neither its position nor any distinguishing mark.  So how can you know whether the next you notice is a new one or one already counted?  By the uncertainty principle, the more closely you pin down its position at one instant, the more uncertain you are of its velocity and where it will turn up next.  When you retire to rest, as a variant to counting sheep in a green field, perhaps you may like to try counting electrons in a probability distribution.  \citep[p. 171-172]{Eddington1939}
\end{quote}


- debunking $N$ as an important/objective number (after giving calculation, he thinks it shows lack of objectivity).  reminds me of Poincare: not objective, not arbitrary, but a convention imposed

\begin{quote}
    I have told you what I believe to be the true story of the cosmical number $N$.  What should we conclude from it?

    To put it crudely, we have \emph{debunked} $N$.  It is not an enumeration of a crowd of discrete particles constituting the objective universe.  Since it is merely a number foisted on us by quantum theory, being associated in an \emph{a priori} way with its methods of analysis, is it any longer of interest?  I do not think its scientific interest is at all affected.  Intrinsically the number of particles in the universe, even if it were genuine, would be a matter of rather trivial curiosity.  The number is scientifically iimportant because it keeps cropping up in more prosaic problems.  It fixes the ratio of the electrical to the gravitational force between a proton and electron---a quantity which practical physicists have been at great pains to determine.  \citep[p. 177]{Eddington1939}
\end{quote}

\begin{quote}
    I have singled out $N$ for attention because, of all the knowledge comprised in fundamental physics, knowledge of the number of elementary particles had seemed least likely to be tainted with subjectivity.  It was therefore particularly suitable for a test case.  But the same subjectivity appears everywhere and is usually not so difficult to discern.  The whole scheme of physical law is debunked, if you like to put it that way.  But debunking the laws of optics will not put out the sun's light; debunking the law of gravitation will not prevent us from falling down stairs; debunking the laws of ballistics will not put a stop to war.  Even if the mystery is torn from them, the laws of our semi-subjective universe are valid in that universe, and in the technical discoveries and inventions of science will continue to bear fruit for good or evil.  \citep[p. 178]{Eddington1939}
\end{quote}



\begin{quote}
    Since then microscopic physics has made great progress, and its laws have turned out to be comprehensible to the mind; but, as I have endeavoured to show, it also turns out that they have been imposed by the mind---by our forms of thought---in the same way that the molar laws are imposed. \citep[p. 180]{Eddington1939}
\end{quote}



- applicability of ``laws of chance'' vs. Heisenberg uncertainty limits and quantum probabilities??? !!!

\begin{quote}
    In current physical theory the undetermined element in the behaviour of a system is treated as a matter of chance.  If there were serious deviations from the law of chance, observation and theory would not agree.  We may therefore say that it is a hypothesis in physics, supported by observation, that there are no objective laws of governance---unless chance is described as a law.

    Nevertheless, if we take a wider view than that of physics, I think it would be misleading to regard chance as the characteristic feature of the objective world.  The denial of objective laws of governance is not so much a hypothesis of physics as a limitation of its subject matter.  Deviations from chance occur, but they are regarded as manifestations of something outside physics, namely consciousness or (more debatably) life.  There is in a human being some portion of the brain, perhaps a mere speck of brain-matter, perhaps an extensive region, in which the physical effects of his volitions begin, and from which they are propagated to the nerves and muscles which translate the volition into action.  We will call this portion of the brain-matter ``conscious matter''.  It must be exactly like inorganic matter in its obedience to the fundamental laws of physics which, being of epistemological origin, are compulsory for all matter; but it cannot be identical in all respects with inorganic matter, for that would reduce the body to an automaton acting independently of consciousness.  The difference must necessarily lie in the undetermined part of the behaviour; the part of the behaviour which is undetermined by the fundamental laws of physics must in conscious matter be governed by objective law or direction instead of being wholly a field of chance.  

    The term ``law of chance'' tends to mislead, because it is applied to what is merely an absence of law in the usual sense of the term.  It is clearer to describe the conditions by reference to correlation.  The hypothesis of current physical theory, which is confirmed by observation of inorganic phenomena, is that there is no correlation of the undetermined behaviour of the individual particles.  

    Accordingly the distinction between ordinary matter and conscious matter is that in ordinary matter there is no correlation in the undetermined parts of the behaviours of the particles, whereas in conscious matter correlation may occur.  Such correlation is looked upon as an interference with the ordinary course of nature, due to the association of consciousness with the matter; in other words, it is the physical aspect of a volition.  This does not mean that, in order to execute a volition, consciousness must direct each individual particle in such a way that correlation occurs.  The particles are merely a representation of our knowledge in the frame of thought corresponding to the concept of analysis and the atomic concept.  When we apply the system of analysis which gives this representation, we cannot foresee whether the resulting particles will have correlated or uncorrelated behaviour; that depends entirely on the objective characteristics of whatever it is that we are analysing.  When non-correlation is assumed, as is customary in physics, it is assumed as a hypothesis.  But, without making any hypothesis, we can say that correlation and non-correlation are representations in our frame of thought of different objective characteristics; and since non-correlation admittedly represents the objective characteristic of systems to which the ordinary formulae of physics apply, correlation must represent another objective characteristic which---since it is not characteristic of systems to which the formulae of physics apply---is regarded by us as something ``outside physics''.

    In the discussion of freewill provoked by the modern physical theories, it has, I think, generally been assumed that, since the ordinary laws of inorganic matter leave its behaviour undetermined within a certain narrow range, there can be no scientific objection to allowing a volition of consciousness to decide the exact behaviour within the limits of the aforesaid range.  I will call this hypothesis $A$.  For any system on a molar scale the permitted range is exceedingly small; and very far-fetched suppositions are necessary to enable volition, working in so small a range, to produce large muscular movements.  To obtain a wider range we must admit correlation of the behaviour of the particles.  This is the theory we have been discussing, and will be called hypothesis $B$.  In former writings I have advocated hypothesis $B$ mainly on the ground of the inadequacy of hypothesis $A$; but in the present mode of approach hypothesis $B$ presents itself as the obvious and natural solution.  

    Although leading to the same conclusion, my earlier discussions\footnote{\emph{The Nature of the Physical World}, pp. 310-315.  \emph{New Pathways in Science}, p. 88.} were marred by a failure to recognise that hypothesis $A$ is nonsense; so that I was more apologetic than I need have been for going beyond it.  There is no half-way house between random and correlated behaviour.  Either the behaviour is wholly a matter of chance, in which case the precise behaviour within the Heisenberg limits of uncertainty depends on chance and not on volition.  Or it is not wholly a matter of chance, in which case the Heisenberg limits, which are calculated on the assumption of non-correlation, are irrelevant.  If we apply the law of chance to the tossing of a coin, the number of heads in 1000 throws is undetermined within the limits, say, 450 to 550.  But if a coin-tossing machine is used which picks up and throws the coin not entirely at random, the non-chance element is not a factor deciding which number between 450 and 550 will turn up; a correlation, or systematic tendency in tossing, may produce any number of heads from 0 to 1000.  

    The fallacy of hypothesis $A$ was that it assumed the behaviour to be restricted by the ordinary laws of physics including the hypothesis of non-correlation or ``law of chance'', and then to be further restricted (or decided) by a non-chance factor (volition).  But we cannot suppose the behaviour to be restricted by chance and non-chance (non-correlation and correlation) simultaneously.  The applicability of the law of chance is a hypothesis; the admission that the behaviour is not governed solely by chance denies the hypothesis.  So if we admit volition at all, we must not forget first to remove the hypothesis of chance if we have been applying it; in particular we must drop the Heisenberg limits which apply only to non-correlated behaviour.  If volition operates on the system, it does so without regard to the Heisenberg limits.  Its only limits are those imposed by the fundamental epistemological laws.  

    Our volitions are not entirely unconsequential; so that there must be laws of some kind applying to them and connecting them with other constituents of consciousness, though such laws are not expected to be of the mathematically exact type characteristic of subjective law.  Primarily the sphere of objective law is the interplay of thoughts, emotions, memories and volitions in consciousness.  In controlling volitions objective law controls also the correlations which are the physical counterparts of volitions.  

    Our philosophy has led to the view that in so far as we can separate the subjective and objective elements in our experience, the subjective is to be identified with the physical and the objective with the conscious and spiritual aspects of experience.  To this we now add, as a helpful analogy provided it is not pressed too far, that conscious purpose is the ``matter'' and chance the ``empty space'' of the objective world.  In the physical universe matter occupies only a small region compared with the empty space; but, rightly or wrongly, we look on it as the more significant part.  In the same way we look on consciousness as the significant part of the objective universe, though it appears to occur only in isolated centres in a background of chaos.  \citep[p. 180-184]{Eddington1939}
\end{quote}

- quoting section IV in total, ``rational correlation of experience'', orthodoxy seeming to pay lip service to [idealist position] without acknowledging or coming to grips with major implications [anti-realism], epistemological confusion, makes clear that subjective view applies in particular to physics

- maybe he would like epistemic psi?

\begin{quote}
    I am about to turn from the scientific to the philosophical setting of scientific epistemology.  This is accordingly a suitable place at which to make a comparison with the most commonly accepted view of scientific philosophy.  The following statement is fairly typical:

    \begin{quote}
        That science is concerned with the rational correlation of experience rather than with the discovery of fragments of absolute truth about an external world is a view which is now widely accepted.\footnote{Unsigned review, \emph{Phil. Mag.}, vol. 25, p. 814, 1938}
    \end{quote}

    I think that the average physicist, in so far as he holds any philosophical view at all about his science, would assent.  The phrase ``rational correlation of experience'' has a savour of orthodoxy which makes it a safe gambit for applause.  The repudiation of more adventurous aims gives a comfortable feeling of modesty---all the more agreeable if we fancy that someone else is being told off.  For my own part I accept the statement, provided that ``science'' is understood to mean ``physics''.  It has taken me nearly twenty years to accept it; but by steady mastication during that period I have managed to swallow it all down bit by bit.  Consequently I am rather flabbergasted by the light-hearted way in which this pronouncement, carrying the most profound implications both for philosophy and for physics, is commonly made and accepted.  

    I have no serious quarrel with the average physicist over his philosophical creed---except that he forgets all about it in practice.  My puzzle is why a belief that physics is concerned with the correlation of experience and not with absolute truth about the external world should usually be accompanied by a steady refusal to treat theoretical physics as a description of correlations of experience and an insistence on treating it as a description of the contents of an absolute objective world.  If I am in any way heterodox, it is because it seems to me a consequence of accepting the belief, that we shall get nearer to whatever truth is to be found in physics by seeking and employing conceptions suitable for the expression of correlations of experience instead of conceptions suitable for the description of an absolute world. 

    The statement evidently means that the methods of physics are incapable of discovering fragments of absolute truth about an external world; for we should have no right to withhold from mankind the absolute truth about the external world if it were within our reach.  If the laboratories, built and endowed at great expense, could assist in the discovery of absolute truth about the external world, it would be reprehensible to discourage their use for this purpose.  But the assertion that the methods of physics cannot reveal absolute (objective) truth or even fragments of absolute truth, concedes my main point that the knowledge obtained by them is wholly subjective.  Indeed it concedes it far too readily; for the assertion is one that ought only to be made after prolonged investigation.  As I have pointed out, sciences other than physics and chemistry are not so limited in their scope.  The discovery of unmistakable signs of intelligent life on another planet would be hailed as an epoch-making astronomical achievement; it can scarcely be denied that it would be the discovery of a fragment of absolute truth about the world external to us.

    Keeping to physics, the commonly accepted scientific philosophy is that it is not concerned with the discovery of absolute truth about the external world, and its laws are not fragments of absolute truth about the external world, or, as I have put it, they are not laws of the objective world.  What then are they, and how is it that we find them in our correlations of experience?  Until we can see, by an examination of the procedure of correlation of our observational experience, how these highly complex laws can have got into it subjectively, it seems premature to accept a philosophy which cuts us off from all other possible explanations of their origin.  This is the examination that we have been conducting.

    The end of our journey is rather a bathos after so much toil.  Instead of struggling up to a lonely peak, we have reached an encampment of believers, who tell us ``That is what we have been asserting for years''.  Presumably they will welcome with open arms the toilworn travellers who have at last found a resting place in the true faith.  All the same I am a bit dubious about that welcome.  Perhaps the assertion, like many a religious creed, was intended only to be recited and applauded.  Anyone who \emph{believes} it is a bit of a heretic.  \citep[p. 184-186]{Eddington1939}

    
\end{quote}

- last comment reminds me of J.S. Mill talking about how it isn't enough to coincidentally believe the truth (living truth vs. dead dogma).  Perhaps he was aware of On Liberty or had to read Mill.


\subsection{ChapterXII: The Beginnings of Knowledge}

- FINALLY HE MENTIONS KANT, VERY RESTRICTED verification principle/instrumentalism of logical positivism

\begin{quote}
    It is, I think, inadvisable to try to describe a scientifically grounded philosophy by the labels of the older philosophical systems.  To accept such a label would make the scientist a party to controversies in which he has no interest, even if he does not condemn them as altogether meaningless.  But if it were necessary to choose a leader from among the older philosophers, there can be no doubt that our choice would be Kant.  We do not accept the Kantian label; but, as a matter of acknowledgement, it is right to say that Kant anticipated to a remarkable extent the ideas to which we are now being impelled by the modern developments of physics.  

    Reference may also be made to another general philosophical system, namely \emph{logical positivism}.  Our insistence that physical quantities are to be defined in such a way that the assertions of physics admit of observational verification, may suggest an affinity with logical positivism.  The meaning of a scientific statement is to be ascertained by reference to the steps which would be taken to verify it.  This will be recognised as a tenet of logical positivism---only it is there extended to all statements.  When it is limited, as here, to items of physical knowledge, it is in no sense a philosophical tenet; it is only a bringing into line of the language of theoretical and of experimental physics, so that we may not claim the support of observation for assertions which have no observational foundation.  If it were a general characteristic of knowledge, it would not be so useful to us in discriminating physical knowledge from other kinds of knowledge.  We are therefore not particularly predisposed to favour the more general assertion of logical positivism that the meaning of all non-tautological statements is to be ascertained in the same way, namely by reference to the procedure of verifying them.  \citep[p. 188-189]{Eddington1939}
\end{quote}


- leans into phil mind and phenomenology in section II, subjective or sympathetic knowledge of a subject (as opposed to structural k.), basically knowledge/theory of other minds in social complex, refs Wells's Country of the Blind and justifiably rejecting others sense knowledge, talks about solipsism and consciousness, 



\begin{quote}
    Without the sympathetic faculty which enables me to recognise myself, not as an individual \emph{mei generis}, but as an element of a social complex, the conception of ``human knowledge'' could not arise; and it would therefore seem illogical to reject this faculty in defining the extent of human knowledge.  

    No one believes in solipsism, and very few even assert that they do.  Those who are obsessed by the word ``existence'' come somehow to the conclusion that other consciousnesses besides their own exist; that is to say, other consciousnesses can be the subject of that mysterious sentence which they never finish.  Those who adopt the epistemological approach take for their subject matter a knowledge which embodies the experiences of other individuals on the same footing as their own experience.  Formally this is non-committal; it is not necessary to assign reasons for choosing a particular theme of study.  But undoubtedly the choice is determined by a conviction, akin to religious conviction, that this co-operative knowledge is the most worth while.  This conviction is inconsistent with a solipsistic outlook.

    It would be meaningless to attribute consciousness to another man without knowing at all what we are attributing to him. But consciousness is not a structural concept describable by purely structural knowledge; nor is the consciousness that we attribute to another man anything of which we have direct awareness, since it is not our own consciousness.  It follows that, if our recognition of conscious beings other than ourselves has any meaning at all, their consciousness must be something of which we have a knowledge which is neither structural knowledge nor direct awareness; and any description of it must be expressed in terms of the third kind of knowledge which we have called sympathetic understanding.  \citep[p. 193-194]{Eddington1939}
\end{quote}

- then he basically talks about qualia and how the physicist ``has a domain independent of sympathetic knowledge'', he notes that he wants to avoid  ``the dilemma of either  (a) denying that there is any knowledge other than physical knowledge or (b) relapsing into the solipsism which we repudiate at the very beginning of physical science.'' (p. 195)

- sensations and feelings, 

\begin{quote}
    The recognition that certain memories are to be treated as a knowledge of past sensations is essential for physical science; because, as we shall see later, the first step towards structural knowledge is a comparison of sensations in one consciousness.  The datum of physical science is not awareness of a sensation, but awareness that a sensation is like, or different from, a sensation which we formerly had.  Granting this, the sensations of one person alone provide sufficient material for structural analysis; and it would be possible to develop from it a scientific theory which, except that it is presented in an egocentric frame of thought, would agree with ordinary physical theory.  But since the analysis would never take us outside a single consciousness, it would give no indication of a world external to that consciousness.  The externality of the physical world results from the fact that it is made up of structures found in different consciousnesses.
    
    Thus the recognition of sensations other than our own, though not required until a rather later stage of the discussion, is essential to the derivation of an \emph{external} physical universe.  Our direct awareness of certain aural and visual sensations (words heard and read) is postulated to be an indirect knowledge of quite different sensations (described by the words heard and read) occurring elsewhere than in our own consciousness.  Solipsism would deny this; and it is by accepting this postulate that physics declares itself anti-solipsistic.  \citep[p. 198-199]{Eddington1939}
\end{quote}


- awareness

\subsection{ChapterXIII: The Synthesis of Knowledge}

- pointing, phil. lang., inferences from data not linguistic pointers of data

- ``I know that'' is idempotent (?)

- holism about consc.

\begin{quote}
    To sum up: ``I'' is first a label or pointer-word attached to a particular consciousness, and consequentially to the sensations, emotions, etc. into which the consciousness is divided by the concept of analysis; and secondly, as associated with self-consciousness, it is part of a verbal form ``I am aware of `I''' used to point to a residuum of awareness which eludes the concept of analysis.  The phrase points to the datum (of which we have immediate knowledge) that our whole awareness is not fully represented by the parts into which we customarily divide it; in other words, it is a unity and not an assemblage of parts.  It appears to be no more than linguistic custom that ``I'' is made in the first case the subject and in the second case the object of the verb ``to be aware''.  When we try to get behind the wording, we find nothing to support the view that awareness is a subject-object relation or even a subject-intransitive relation.  \citep[p. 206-207]{Eddington1939}
\end{quote}

- phil lang critique of realism

\begin{quote}
    And, so I suppose, realists will not insist that whenever we gesticulate something must be gesticulated, and that the something is unaffected by our gesticulation of it, being indeed precisely what it would be if it were not being gesticulated.  Yet I sometimes wonder how a realist would regard the gesture known as ``cocking a snook''.  It would seem clear that something must be cocked; and I fear the only logical conclusion is that there is a realm of existence containing uncocked snooks which are exactly what they would be if they were being cocked---but perhaps that is too dangerous a thought to pursue when philosophers are trying to express what they think of one another!  \citep[p. 213-214]{Eddington1939}
\end{quote}

\begin{quote}
    It would seem that the first time we perceive a new taste, our consciousness becomes modified in such a way that thereafter an imagining of the taste is possible.  We ordinarily say that a memory of the taste is stored up in it.  I do not see how this can be reconciled with the realist view that imagining and perceiving are independent relations of consciousness to a sensum outside consciousness.  
    
    In the passage that I have quoted it is recognised that, if perceiving is purely a relation between the mind and an external object, the object is not modified by our perceiving of it.  It is not clear whether it is also recognised that the mind is not modified.  If the mind is modified by the act of perceiving, it is incorrect to describe perceiving as a ``relation''; and the argument based on the existence of more than one kind of relation falls to the ground.  On the other hand, if neither the mind nor the sensum is modified by the act of perceiving, how is it that it is not until after the perception that a new kind of relation of the mind to the sensum becomes possible, namely remembering or imagining?  \citep[p. 215]{Eddington1939}
\end{quote}

\begin{quote}
    It would be illogical to attribute the similarity of the structures in different consciousnesses to a common cause without allowing to the common cause a status fully as objective as the structures themselves.  I therefore take it as axiomatic that the external world must have objective content.  But according to our conclusions, the laws of physics are a property of the frame of thought in which we represent our knowledge of the objective content, and thus far physics has been unable to discover any laws applying to the objective content itself.  This raises the question, How is it that we are able to make successful predictions of phenomena without knowing any law controlling the objective content of the universe and therefore without knowing how the objective content is going to behave?

    Although it is rather the fashion for scientific writers to say that physics is not concerned with objective truth, it would be unsafe to take them at their word.  Apparently the statement is intended to closure discussion, rather than to assert a principle whose far-reaching implications invite investigation.  Our own conclusion is less sweepingly expressed; but it is meant seriously, and we must examine the difficulties to which it seems to lead.  

    Much of the difficulty disappears if we keep in mind that \emph{pure} subjectivity is confined to the laws---the regularities---of the physical world.  The variety of appearances around us is primarily an objective variety.  That a subjective distortion is introduced in our apprehension of things is no more than physicists have been accustomed to admit.  \citep[p. 217]{Eddington1939}
\end{quote}


\begin{quote}
    It is often pointed our that the primary difference of outlook between the scientist and the savage is that the savage attributes all that he finds mysterious in nature to the activity of demons or other spirits.  For the savage any physical object may be possessed of demonic volition, and it is impossible to count on its behaviour except in so far as the directing demon may be amenable to prayer and propitiation.  Physical science has made a place for itself by greatly limiting the sphere of demonic activity, so that there is an extensive realm of experience in which behaviour can be counted on and scientific prediction is possible.  Great as may be the practical effects of this change, it is a matter of detail (special fact) rather than of principle.  Demonic activity (volition) remains, though it is limited to certain centres in men and the higher animals.  Prayer and propitiation may still influence the course of physical phenomena when directed to these centres.  We now think it ludicrous to imagine that rocks, sea and sky are animated by volitions such as we are aware of in ourselves.  It would be thought even more ludicrous to imagine that the volitionless behaviour of rocks, sea and sky extends also to ourselves, were it not that we have scarcely yet recovered from the repressions of 250 years of deterministic physics.  

    Accordingly we do not regard the principle of non-correlation as one of the fundamental laws of physics.  Non-correlation usually applies; but correlation occurs exceptionally, and the result is an unexpectedness of behaviour which is recognised by us as a physical manifestation of conscious volition.  In saying that the behaviour is unexpected, we mean unexpected from the point of view of physics, which supplies the gap left by our ignorance of the springs of objective behaviour by assuming non-correlation.  Actually the volitional behaviour may be fully expected---it may be an answer to our own request---but this expectation takes into account knowledge of the objective world not comprised in physical science and not reducible to the accepted pattern of physical law.  In so far as the comparative rarity of correlation can be considered a law, it is a law of distribution of consciousness rather than a law of the physical world.  \citep[p. 219-220]{Eddington1939}
\end{quote}


\section{Kilmister on Eddington}

- ch. 1, mystery of Eddington (great astrophysicist) being more speculative and perhaps wrong in two later books in period after Dirac electron (1928)

- ch. 2, Einstein special relativity (Galilean to Lorentzian transformations, rejection of Newtonian absolute time) not liked by British Maxwellians

\begin{quote}
    Such is the real content of Einstein's paper but this is not quite how things looked when seen from Cambridge.  The influence of Maxwell himself had been profound.  He had inferred from his electromagnetic equations the existence of wave solutions travelling with the speed of light and he had identified light itself with such a solution.  The problem for Maxwell, and many British physicists, was the medium of transmission of the waves.  This supposed medium, the aether, was seen by Maxwell as a physically real diffuse medium filling the universe.  The definitive account of this at the turn of the century was \emph{Aether and Matter} (Larmor 1900).  In it, aether theory had assumed a very subtle and elegant form.  But if the aether was physical it should be possible to determine the motion of the earth through it by optical experiments.  Maxwell suggested such an experiment, which was carried out in America by Michelson, first alone and then in conjunction with Morley.  
    
    In the Michelson-Morley experiment a beam of light is split in two by a half-silvered mirror.  The two parts travel out and back along two directions at right-angles, being reflected by mirrors at the ends of two rigid arms.  The relative time of the return rays is monitored by means of optical `interference'.  A patter of light and dark bands is obtained on a screen, because the two paths do not differ by an exact number of wavelengths.  The apparatus is then rotated in the same horizontal plane through a right-angle.  If one arm were, for instance, initially moving forward through the aether, so that the light speed would be different from that along the other transverse arm, then after rotation the situation will be reversed.  The first arm will now be transverse to the `aether wind' and the other arm will be suffering the effect of opposite movement.  The interference fringes will therefore shift.  No such effect was found, nor was this a coincidence resulting from the earth just happening to be at rest in the aether at the time.  For a repetition six months later again gave a null result.
    
    Einstein makes no reference to the Michelson-Morley experiment in his paper and at one time believed himself to have been ignorant of it, but later he admitted that it must have had some unconscious effect on his thinking.  In the very much more aether-conscious atmosphere of British physics, the experiment was seen as an empirical demonstration that it was impossible to detect the motion of the earth through the aether.  This led to a variety of responses.  Few were yet happy to draw the obvious conclusion that, unfortunate as it might be for any intuitive need to have a medium in which light was travelling, no such aether existed.  Some found Einstein's explanation of the Michelson-Morley experiment (that, as a simple consequence of the Lorentz transformation, the speed of light, unlike all other speeds, is unchanged by the motion of the observer) uncongenial and preferred to explain the result by hypothesising changes in the lengths of the arms of the apparatus.  At the extreme of this party can, perhaps, be placed Rutherford who, as late as 1910 rejoined, to the suggestion that no Anglo-Saxon could understand relativity, `No! they have too much sense'.  Between the extremes were those who accepted Einstein's theory with greater or less enthusiasm, and saw as the next problem the rendering of it and aether theory as a consistent whole.
    
    \citep[p. 16-18]{Kilmister1994}
\end{quote}